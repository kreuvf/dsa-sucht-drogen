1 & 1 & Trägheit (körperlich) & - & 
{\begin{itemize}[nosep]
\item \vspace*{-\baselineskip}1W4 körperliche Eigenschaften sinken um je 1W6
\item AT-, PA-, FK-Basiswert sinken um je 1W6
\item GS sinkt um 1W3\vspace*{-\baselineskip}
\end{itemize}} & ✔ & ✔ & ✔ & ✔ \\
1 & 2 & Trägheit (geistig) & - & 
{\begin{itemize}[nosep]
\item \vspace*{-\baselineskip}1W4 geistige Eigenschaften sinken um je 1W6
\item Initiative-Basis sinkt um 1W6\vspace*{-\baselineskip}
\end{itemize}} & ✔ & ✔ & ✔ & ✔ \\
1 & 3 & Schmerzen (stechend, brennend, scharf oder diffus begrenzt, kribbelnd, klopfend/pochend) & 2W6 & 
{\begin{itemize}[nosep]
\item \vspace*{-\baselineskip}\emph{Selbstbeherrschungs}-Probe um Nebenwirkungswert erschwert zum Ignorieren im Kampf und außerhalb (\~{}1 SR pro Probe)
\item Im Kampf: Einschränkungen wie Wunde an betroffenem Körperteil ohne Schadenspunkte
\item Außerhalb des Kampfes:\begin{itemize}[nosep]
\item 1W20-Proben um halben Nebenwirkungswert erschwert
\item 3W20-Proben um Nebenwirkungswert erschwert
\end{itemize}\vspace*{-\baselineskip}
\end{itemize}}
 & ✔ & ✔ & ✔ & ✔ \\
1 & 4 & Juckreiz & 7~+~1W6 & 
{\begin{itemize}[nosep]
\item \vspace*{-\baselineskip}\emph{Selbstbeherrschungs}-Probe um Nebenwirkungswert erschwert zum Ignorieren im Kampf und außerhalb (\~{}1 SR pro Probe)
\item vor Proben mit 1W20 gegen Nebenwirkungswert würfeln
\begin{itemize}[nosep]
\item gelungen: nächste Probe um Nebenwirkungswert erschwert
\item misslungen: nächste Probe um halben Nebenwirkungswert erschwert
\end{itemize}\vspace*{-\baselineskip}
\end{itemize}}
 & ✔ & ✔ & ✔ & ✔ \\
1 & 5 & allergische Symptome:
{\begin{itemize}[nosep]
\item Augen, Nase zugeschwollen
\item tränende Augen mit Fremdkörpergefühl
\item laufende Nase\vspace*{-\baselineskip}
\end{itemize}} & - & 
{\begin{itemize}[nosep]
\item \vspace*{-\baselineskip}Proben auf \emph{Sinnenschärfe (Sehen)} und \emph{Sinnenschärfe (Riechen)} um je 1W6 erschwert; vor jeder Probe erneut auswürfeln
\item Regeneration halbiert
\item Erschöpfung baut sich nur auf die Hälfte des Maximalwertes ab\vspace*{-\baselineskip}
\end{itemize}} & ✔ & ✔ & ✔ & ✔ \\
1 & 6 & Krämpfe & - & 
{\begin{itemize}[nosep]
\item \vspace*{-\baselineskip}GE und FF sinken um je 1W6
\begin{itemize}[nosep]
\item je höher der Wert, desto mehr Körperteile sind betroffen
\end{itemize}
\item GS sinkt um 1W6\vspace*{-\baselineskip}
\end{itemize}} & ✔ & ✔ & ✔ & ✔ \\
1 & 7 & Regenerationsstörung & - & 
{\begin{itemize}[nosep]
\item \vspace*{-\baselineskip}nächtliche Regeneration (Lebensenergie und Astralenergie) sinkt um 1W6\vspace*{-\baselineskip}
\end{itemize}} & ✔ & ✔ & ✔ & ✔ \\
1 & 8 & Reaktionsfähigkeit eingeschränkt & - & 
{\begin{itemize}[nosep]
\item \vspace*{-\baselineskip}PA-, Initiative-Basis sinkt um je 1W6
\item Wirkung der Sonderfertigkeiten \emph{Aufmerksamkeit}, \emph{Kampfgespür} und \emph{Kampfreflexe} aufgehoben\vspace*{-\baselineskip}
\end{itemize}} & ✔ & ✔ & ✔ & ✔ \\
1 & 9 & Anschwellen von Körperteilen (Wassereinlagerungen) & - & 
{\begin{itemize}[nosep]
\item \vspace*{-\baselineskip}GE und FF sinken um je 2W6
\item GS sinkt um 1W3
\item entsprechende Kleidungsstücke und Rüstungsteile passen nicht mehr\vspace*{-\baselineskip}
\end{itemize}} & ✔ & ✔ & ✔ & ✔ \\
1 & 10 & eingeschränkter Sinn (1W10) & 2W6 & 
{\begin{itemize}[nosep]
\item \vspace*{-\baselineskip}betroffenen Sinn bestimmen:
\begin{enumerate*}[nosep, itemjoin={{; }}]
\item Hören
\item Sehen
\item Tasten
\item Riechen
\item Schmecken
\item Sprechen
\item Hunger
\item Durst
\item Wärme
\item Kälte
\end{enumerate*}
\item \emph{Sinnenschärfe}-Proben auf den jeweiligen Sinn sind erschwert um den Nebenwirkungswert
\item Alternative: passende Vorteile (herausragender Sinn, Entfernungssinn, etc.) können entzogen, passende Nachteile (Nachtblindheit, Farbenblindheit, etc.) vergeben werden\vspace*{-\baselineskip}
\end{itemize}} & ✔ & ✔ & ✔ & ✔ \\
1 & 11 & empfindlicher Sinn (1W6) & - & 
{\begin{itemize}[nosep]
\item \vspace*{-\baselineskip}betroffenen Sinn bestimmen
\begin{enumerate*}[nosep, itemjoin={{; }}]
\item Hören
\item Sehen
\item Tasten
\item Riechen
\item Schmecken
\item Wärme/Kälte
\end{enumerate*}
\item Überempfindlichkeit analog zu entsprechenden Nachteilen\vspace*{-\baselineskip}
\end{itemize}} & ✔ & ✔ & ✔ & ✔ \\
1 & 12 & Lichtscheu/lichtempfindlich & - & 
{\begin{itemize}[nosep]
\item \vspace*{-\baselineskip}Der Betroffene erhält den Nachteil \enquote{lichtscheu}.
\item Verfügt der Betroffene bereits über den Nachteil \enquote{lichtscheu}, so erhält er den Nachteil \enquote{lichtempfindlich}
\item Verfügt der Betroffene bereits über den Nachteil \enquote{lichtempfindlich}, so wird die Stärke der Auswirkungen verdoppelt.\vspace*{-\baselineskip}
\end{itemize}} & ✔ & ✔ & ✔ & ✔ \\
1 & 13 & Hustenreiz und Bluthusten & - & 
{\begin{itemize}[nosep]
\item \vspace*{-\baselineskip}1W20-Proben erschwert um 1W6; vor jeder Probe erneut auswürfeln
\item 3W20-Proben erschwert um 2W6; vor jeder Probe erneut auswürfeln
\item gescheiterte Proben bedeuten, dass der Betroffene 2W3 Aktionen mit Husten/Auswurf beschäftigt ist\vspace*{-\baselineskip}
\end{itemize}} & ✔ & ✔ & ✔ & ✔ \\
1 & 14 & Atemnot & - & 
{\begin{itemize}[nosep]
\item \vspace*{-\baselineskip}Regeneration halbiert
\item Ausdauer auf Hälfte des Maximums begrenzt
\item Erschöpfung baut sich doppelt so schnell auf\vspace*{-\baselineskip}
\end{itemize}} & ✔ & ✔ & ✘ & ✘ \\
1 & 15 & Herzrasen & - & 
{\begin{itemize}[nosep]
\item \vspace*{-\baselineskip}Betroffener verfällt in Panik
\item 2W6 Erschöpfung pro Tag\vspace*{-\baselineskip}
\end{itemize}} & ✔ & ✔ & ✘ & ✘ \\
1 & 16 & Erbrechen & - & 
{\begin{itemize}[nosep]
\item \vspace*{-\baselineskip}1W6 Erschöpfung pro Tag
\item 1W20-Proben erschwert um 1W6 − 4; vor jeder Probe erneut auswürfeln
\item 3W20-Proben erschwert um 2W6 − 8; vor jeder Probe erneut auswürfeln\vspace*{-\baselineskip}
\end{itemize}} & ✔ & ✔ & ✔ & ✘ \\
1 & 17 & Durchfall & - & 
{\begin{itemize}[nosep]
\item \vspace*{-\baselineskip}1W6 Erschöpfung pro Tag
\item Proben müssen doppelt ausgewürfelt werden; das schlechtere Ergebnis zählt
\begin{itemize}[nosep]
\item Ablenkung durch Durchfall kann ignoriert werden (es geht in die Hose), kann aber leicht Ansehen kosten oder zu neuen Spitznamen führen\vspace*{-\baselineskip}
\end{itemize}
\end{itemize}}
 & ✔ & ✔ & ✔ & ✘ \\
1 & 18 & harntreibende oder antidiuretische Wirkung (1W2) & - & 
{\begin{enumerate}[nosep]
\item \vspace*{-\baselineskip}harntreibend: Betroffener verspürt nach jedem Wasserlassen binnen 1W6 Spielrunden erneut Harndrang und muss den Flüssigkeitsverlust ausgleichen
\item antidiuretisch: Betroffener verspürt keinen Harndrang mehr, da keiner gebildet wird. Dies führt über kurz oder lang zur Anreicherung von Ausscheidungsprodukten, sodass die KO und die Regeneration pro Tag um 1 fallen. Ab dem 7. Tag erleidet der Betroffene pro Tag 2 SP, bis zum Tod.\vspace*{-\baselineskip}
\end{enumerate}}
 & ✔ & ✔ & ✔ & ✘ \\
1 & 19 & Impotenz & - & 
{\begin{itemize}[nosep]
\item \vspace*{-\baselineskip}betrifft nur männliche Charaktere
\item Unsicherheit
\begin{itemize}[nosep]
\item MU sinkt um 2W3
\item CH sinkt um 1W3
\end{itemize}\vspace*{-\baselineskip}
\end{itemize}}
 & ✔ & ✔ & ✔ & ✔ \\
1 & 20 & kosmetische Einschränkungen (1W6) & - & 
{\begin{enumerate*}[nosep, itemjoin={{; }}]
\item Pocken
\item Pusteln
\item Male
\item Warzen
\item Verfärbungen
\item Ausschlag
\end{enumerate*}
\newline Wirkung entspricht Absenkung des Aussehens um eine Stufe (kein Effekt bei widerwärtigem Aussehen)} & ✔ & ✔ & ✔ & ✔ \\
2 & 1 & unangenehmer Geruch & - & Wirkung des entsprechenden Nachteils & ✔ & ✔ & ✔ & ✔ \\
2 & 2 & veränderte/keine Stimme (1W6) & - & 
{\begin{itemize}[nosep]
\item \vspace*{-\baselineskip}1 bis 5: veränderte, beispielsweise heisere, Stimme (siehe Nachteil \enquote{unangenehme Stimme})
\item 6: keine Stimme mehr\vspace*{-\baselineskip}
\end{itemize}} & ✔ & ✔ & ✔ & ✔ \\
2 & 3 & Haarausfall & - & 
{\begin{itemize}[nosep]
\item \vspace*{-\baselineskip}MU sinkt um 1W3
\item CH sinkt um 2W3\vspace*{-\baselineskip}
\end{itemize}} & ✔ & ✔ & ✔ & ✔ \\
2 & 4 & Haarwuchs & - & 
{\begin{itemize}[nosep]
\item \vspace*{-\baselineskip}CH sinkt um 1W3\vspace*{-\baselineskip}
\end{itemize}} & ✔ & ✔ & ✔ & ✔ \\
2 & 5 & Sprechstörungen & - & 
{\begin{itemize}[nosep]
\item \vspace*{-\baselineskip}wie der Nachteil Sprachfehler
\item die Erschwernis liegt allerdings bei 3 + 1W6\vspace*{-\baselineskip}
\end{itemize}} & ✔ & ✔ & ✔ & ✔ \\
2 & 6 & Koordinationsschwierigkeiten/unsicherer Gang & - & 
{\begin{itemize}[nosep]
\item \vspace*{-\baselineskip}Körperbeherrschungsproben werden für einfachste Bewegungen nötig und um 4 + 1W6 erschwert; vor jeder Probe erneut auswürfeln
\item Balance wird um eine Stufe herabgesetzt in der Reihenfolge Herausragende Balance/Balance/Sonderfertigkeit "Standfest"\vspace*{-\baselineskip}
\end{itemize}} & ✔ & ✔ & ✔ & ✔ \\
2 & 7 & Verwirrung/Desorientierung & 2W6 & Wirkung entspricht \enquote{Große Verwirrung} mit ZfP* in Höhe des Nebenwirkungswerts & ✔ & ✔ & ✔ & ✘ \\
2 & 8 & unkontrolliertes Lachen & 2W6 & 
{\begin{itemize}[nosep]
\item \vspace*{-\baselineskip}\emph{Sinnenschärfe}-Proben erschwert um Nebenwirkungswert
\item Wirkung der Sonderfertigkeiten \emph{Aufmerksamkeit}, \emph{Kampfgespür} und \emph{Kampfreflexe} aufgehoben
\item \emph{Selbstbeherrschungs}-Probe um Nebenwirkungswert erschwert zum Unterdrücken im Kampf und außerhalb (\~{}1 SR pro Probe)
\item Lachen kann ansteckend sein, daher muss jeder, der den Lachenden hören kann, mit einer passenden Probe überprüfen, ob er sich davon anstecken lässt und 1 SR lang mitlacht
\begin{itemize}[nosep]
\item allgemein \emph{Selbstbeherrschungs}-Probe um den halben Nebenwirkungswert erschwert, weitere Modifikationen abhängig vom Charakter\vspace*{-\baselineskip}
\end{itemize}
\end{itemize}}
 & ✔ & ✔ & ✘ & ✘ \\
2 & 9 & Halluzinationen (eine/mehrere/alle Sinne betreffend) & 2W6 & Wirkung entspricht \enquote{Halluzination} mit ZfP* in Höhe des Nebenwirkungswerts & ✔ & ✔ & ✘ & ✘ \\
2 & 10 & Einbildungen/Wahnvorstellungen & - & Wirkung entspricht den gleichnamigen Nachteilen & ✔ & ✔ & ✔ & ✔ \\
2 & 11 & Visionen & - & Die Häufigkeit der Auslösung ist Meisterentscheid. Es gilt: Weniger ist mehr.
{Visionen: Das Bewusstsein des Betroffenen rutscht leicht in einen Zustand, der äußerlich durch beschleunigte Atemfrequenz, hohen Puls und Nichtansprechbarkeit gekennzeichnet ist. Der Körper kann unter starker Anspannung stehen, die Augen sind weit geöffnet, der starre Blick scheint aber nichts zu fixieren, das Außenstehende sehen könnten. Der Betroffene sieht in seinen Visionen Bruchstücke der Vergangenheit, der Gegenwart und aller möglichen Zukünfte, häufig auch überlagert und aufgrund der Rauschmittelwirkung verstörend verzerrt.
\begin{itemize}[nosep]
\item Während eine Vision für den Betroffenen gefühlt nur etwa eine Spielrunde lang anhält, vergehen tatsächlich 1W20 Spielrunden (für jede Vision einzeln auswürfeln).
\item 1W3 Punkte Erschöpfung
\item Zur Verarbeitung des Gesehenen wird eine Stunde benötigt:
\begin{itemize}[nosep]
\item 1W20-Proben um 2 erschwert
\item 3W20-Proben um 6 erschwert\vspace*{-\baselineskip}
\end{itemize}
\end{itemize}} & ✔ & ✔ & ✔ & ✔ \\
2 & 12 & Flashbacks & - & Die Häufigkeit der Auslösung ist Meisterentscheid. Es gilt: Weniger ist mehr.
{Flashbacks: Ohne äußerliche Warnzeichen findet sich der Betroffene bei vollem Bewusstsein in einer bereits durchlebten Situation wieder. Dabei kann er sich auch aus Sicht eines Dritten wahrnehmen und in dieser Traumwelt handeln wie in der Realität. In der Realität verharrt der Betroffene allerdings in seiner Position. Grundsätzlich hinterlassen Flashbacks ein bedrückendes Gefühl und sie enden so schnell wie sie gekommen sind. Gern gaukeln sie dem Betroffenen vor etwas an zurückliegenden Ereignissen ändern und so Fehler wiedergutmachen zu können.
\begin{itemize}[nosep]
\item Flashbacks dauern nur wenige Kampfrunden lang und die erlebte Zeit entspricht der tatsächlich vergangenen.
\item 1 Punkt Erschöpfung\vspace*{-\baselineskip}
\end{itemize}} & ✔ & ✔ & ✔ & ✔ \\
2 & 13 & Déjà-vus  & - & Die Häufigkeit der Auslösung ist Meisterentscheid. Es gilt: Weniger ist mehr.
{Visionen: Das Bewusstsein des Betroffenen rutscht leicht in einen Zustand, der äußerlich durch beschleunigte Atemfrequenz, hohen Puls und Nichtansprechbarkeit gekennzeichnet ist. Der Körper kann unter starker Anspannung stehen, die Augen sind weit geöffnet, der starre Blick scheint aber nichts zu fixieren, das Außenstehende sehen könnten. Der Betroffene sieht in seinen Visionen Bruchstücke der Vergangenheit, der Gegenwart und aller möglichen Zukünfte, häufig auch überlagert und aufgrund der Drogenwirkung verstörend verzerrt.
\begin{itemize}[nosep]
\item Während eine Vision für den Betroffenen gefühlt nur etwa eine Spielrunde lang anhält, vergehen tatsächlich 1W20 Spielrunden (für jede Vision einzeln auswürfeln).
\item 1W3 Punkte Erschöpfung
\item Zur Verarbeitung des Gesehenen wird eine Stunde benötigt:
\begin{itemize}[nosep]
\item 1W20-Proben um 2 erschwert
\item 3W20-Proben um 6 erschwert\vspace*{-\baselineskip}
\end{itemize}
\end{itemize}} & ✔ & ✔ & ✔ & ✔ \\
2 & 14 & Bewusstlosigkeit & 1W6 & 
{\begin{itemize}[nosep]
\item \vspace*{-\baselineskip}Der Betroffene verliert das Bewusstsein.
\item Der Betroffene kann pro Tag eine KO-Probe ablegen, die um den Nebenwirkungwert erschwert ist, um für 1W3 h aus der Bewusstlosigkeit zu erwachen. Bei einer 1 endet diese Nebenwirkung vorzeitig.\vspace*{-\baselineskip}
\end{itemize}} & ✔ & ✔ & ✔ & ✔ \\
2 & 15 & Koma & 2W6 & 
{\begin{itemize}[nosep]
\item \vspace*{-\baselineskip}Der Betroffene verliert das Bewusstsein und reagiert nicht einmal mehr auf intensivste Reize.
\item Der Betroffene kann pro Tag eine KO-Probe ablegen, die um den Nebenwirkungwert erschwert ist, um von einem Koma zur Bewusstlosigkeit zu gelangen.\vspace*{-\baselineskip}
\end{itemize}} & ✔ & ✔ & ✔ & ✘ \\
2 & 16 & Scheintod & 3W6 & 
{\begin{itemize}[nosep]
\item \vspace*{-\baselineskip}Der Betroffene verliert das Bewusstsein und reagiert nicht einmal mehr auf intensivste Reize.
\item Sein Puls ist so schwach, seine Atmung so flach, dass eine \emph{Heilkunde-Wunden}-Probe erschwert um 4 oder eine \emph{Sinnenschärfe}-Probe (Tastsinn) erschwert um den Nebenwirkungswert (aber mindestens 4) gelingen muss, um den Betroffenen nicht für tot zu halten.
\item Der Betroffene kann pro Tag eine KO-Probe ablegen, die um den Nebenwirkungswert erschwert ist, um vom Scheintod zum Koma zu gelangen.\vspace*{-\baselineskip}
\end{itemize}} & ✔ & ✔ & ✘ & ✘ \\
2 & 17 & Schlafwandeln & - & Ähnlich dem Nachteil, allerdings mit drastischeren Auswirkungen:
{\begin{itemize}[nosep]
\item Einsatz von Gewalt, auch Waffengewalt
\item Einsatz von Magie
\item Einsatz von Liturgien
\item Proben im Kampf während des Schlafwandelns sind um 6 erschwert
\item Zauber- und Liturgieproben sind um 3 erschwert (fehlende Sicht kommt hinzu)\vspace*{-\baselineskip}
\end{itemize}} & ✔ & ✔ & ✔ & ✔ \\
2 & 18 & Einschlafschwierigkeiten/Schlaflosigkeit & - & Wirkung entspricht dem Nachteil Schlafstörungen I, kommt allerdings jede Nacht zu tragen
 & ✔ & ✔ & ✔ & ✔ \\
2 & 19 & Aufwachschwierigkeiten & - & 
{\begin{itemize}[nosep]
\item \vspace*{-\baselineskip}Der Betroffene ist schwerer zu wecken und schläft trotz erholsamer Nacht nach dem ersten Aufwachen am Morgen leicht wieder ein.
\item \emph{Sinnenschärfe}-Proben während des Schlafes sind zusätzlich zur normalen Schlaferschwernis um weitere 5 erschwert.\vspace*{-\baselineskip}
\end{itemize}} & ✔ & ✔ & ✔ & ✔ \\
2 & 20 & Albträume & - & 
{\begin{itemize}[nosep]
\item \vspace*{-\baselineskip}Wirkung entspricht dem Nachteil Schlafstörungen I, kommt allerdings jede Nacht zu tragen
\item Die in einem Albtraum angesprochenen Ängste wirken auch noch 6 h nach dem Aufstehen nach:
\begin{itemize}[nosep]
\item Proben auf die angesprochenen Ängste um 5 erleichtert
\end{itemize}\vspace*{-\baselineskip}
\end{itemize}}
 & ✔ & ✔ & ✔ & ✔ \\
3 & 1 & Verstärkung von Ängsten & - & 
{\begin{itemize}[nosep]
\item \vspace*{-\baselineskip}1W2 Ängste steigen um je 1W5
\item Falls der Betroffene über keine oder zu wenige Ängste verfügt, erhält er welche nach Meisterentscheid, allerdings wird 3 zum Würfelwurf für die Höhe addiert.\vspace*{-\baselineskip}
\end{itemize}} & ✔ & ✔ & ✔ & ✔ \\
3 & 2 & Verstärkung anderer schlechter Eigenschaften & - & 
{\begin{itemize}[nosep]
\item \vspace*{-\baselineskip}1W3 schlechte Eigenschaften, die keine Ängste sind, steigen um je 1W4
\item Falls der Betroffene über keine oder zu wenige schlechte Eigenschaften verfügt, erhält er welche nach Meisterentscheid, allerdings wird 3 zum Würfelwurf für die Höhe addiert.\vspace*{-\baselineskip}
\end{itemize}} & ✔ & ✔ & ✔ & ✔ \\
3 & 3 & Erinnerungsverlust ab Wirkbeginn & - & 
{\begin{itemize}[nosep]
\item \vspace*{-\baselineskip}Der Betroffene ist nicht in der Lage sich an irgendetwas zu erinnern, was ihm seit der Einnahme passiert ist.
\item Diese Wirkung ist stets dauerhaft, da die Erinnerungen nicht überdeckt, sondern gar nicht erst angelegt werden.\vspace*{-\baselineskip}
\end{itemize}} & ✔ & ✔ & ✘ & ✘ \\
3 & 4 & Erinnerungsverlust älterer Erinnerungen & 2W6 & 
{\begin{itemize}[nosep]
\item \vspace*{-\baselineskip}Der Betroffene kann sich an bestimmte Dinge nicht mehr erinnern. Dabei kann es sich um regeltechnisch unnütze Erinnerungen wie die Erinnerung an das Aussehen der eigenen Eltern handeln, aber auch um den Verlust von Talent-, Zauberfertigkeits-, Ritualkenntnis- oder Liturgiekenntnispunkten und Sonderfertigkeiten handeln.
\item Mit dem Wirkende dieser Nebenwirkung kehren die verlorenen Erinnerungen Schritt für Schritt zurück: 3 Punkte oder eine Sonderfertigkeit pro Tag.\vspace*{-\baselineskip}
\end{itemize}} & ✔ & ✔ & ✔ & ✔ \\
3 & 5 & Lernschwierigkeiten & - & 
{\begin{itemize}[nosep]
\item \vspace*{-\baselineskip}Der Betroffene hat größte Schwierigkeiten neues Wissen, egal welcher Art, zu behalten.
\item Während sich AP-Kosten nicht ändern, erhöhen sich die Lernzeiten auf das Vierfache.\vspace*{-\baselineskip}
\end{itemize}} & ✔ & ✔ & ✔ & ✔ \\
3 & 6 & Unterschätzung von Gefahren & 2W6 & 
{\begin{itemize}[nosep]
\item \vspace*{-\baselineskip}\emph{Gefahreninstinkt}-Proben sind um den Nebenwirkungswert erschwert
\item Proben zur Einschätzung von Gefahren (Höhen, Temperaturen, Anzahl und Stärke von Gegnern etc.) sind um den halben Nebenwirkungswert erschwert
\begin{itemize}[nosep]
\item gelungen: wie ohne Nebenwirkung
\item misslungen: definitiv eine falsche Einschätzung, das heißt auch, dass \emph{Gefahreninstinkt} häufiger falsch-positiv anschlägt
\end{itemize}\vspace*{-\baselineskip}
\end{itemize}}
 & ✔ & ✔ & ✔ & ✔ \\
3 & 7 & Selbstüberschätzung & - & Der Betroffene überschätzt die eigenen Fähigkeiten bei Weitem und wird so Aufgaben übernehmen (wollen), denen er nicht gewachsen ist. & ✔ & ✔ & ✔ & ✔ \\
3 & 8 & Enthemmung & - & 
{\begin{itemize}[nosep]
\item \vspace*{-\baselineskip}Der Betroffene vergisst seine Hemmungen und verhält sich daher sozial äußerst unangepasst.
\item Je nach Umständen bringt dies Probenmodifikationen von −9 bis +9 auf gesellschaftlich relevante Proben ein.
\item CH sinkt um 2\vspace*{-\baselineskip}
\end{itemize}} & ✔ & ✔ & ✔ & ✘ \\
3 & 9 & übersteigerte Libido & 4~+~1W6 & 
{\begin{itemize}[nosep]
\item \vspace*{-\baselineskip}Der Betroffene stürzt sich quasi auf alles, was nicht bei drei auf den Bäumen ist.
\begin{itemize}[nosep]
\item Nachteil \enquote{Brünstigkeit} auf dem Nebenwirkungswert.
\end{itemize}
\item Der Talentwert in \enquote{Betören} steigt um den Nebenwirkungswert. Auch ohne das Talent \enquote{Betören} kann auf dieses gewürfelt werden, als Talentwert wird der Nebenwirkungswert genommen.
\item Es fällt dem Betroffenen schwer sich auf andere Dinge zu konzentrieren als die Befriedigung seines Sexualtriebs. Der Betroffene erhält daher den Nachteil \enquote{Unstet}, allerdings sind sämtliche Proben bereits ab Überschreitung einer Dauer von 1 h von der Erschwernis betroffen.\vspace*{-\baselineskip}
\end{itemize}} & ✔ & ✔ & ✔ & ✘ \\
3 & 10 & Fressattacken & 6~+~1W6 & 
{\begin{itemize}[nosep]
\item \vspace*{-\baselineskip}Der Betroffene wird von Fressattacken übermannt, die dazu führen, das er in kurzer Zeit mindestens eine Tagesration zu sich nimmt.
\item \emph{Selbstbeherrschungs}-Probe um Nebenwirkungswert erschwert zum Unterdrücken im Kampf und außerhalb (\~{}6 SR pro Probe)
\item Wird der Betroffene gestört oder abgelenkt, sodass die Fressattacke unterbrochen werden muss, kann dieser jähzornig werden und dann versuchen die Störungsquelle abzustellen.
\begin{itemize}[nosep]
\item Probe auf Jähzorn erleichtert um den Nebenwirkungswert
\item Falls kein Jähzorn vorhanden, erhält der Betroffene Jähzorn auf einem Wert in Höhe des Nebenwirkungswerts
\end{itemize}
\item Nach einem Monat mit dieser Nebenwirkung erhält der Charakter den Nachteil \enquote{fettleibig}.\vspace*{-\baselineskip}
\end{itemize}} & ✔ & ✔ & ✔ & ✔ \\
3 & 11 & Bewegungsdrang & 2W6 & 
{\begin{itemize}[nosep]
\item \vspace*{-\baselineskip}Der Betroffene leidet unter ausgeprägtem Bewegungsdrang.
\item Selbst während Ruhepausen, muss dieser sich ständig bewegen. Während dem Einschlafen wackelt er ständig mit Armen oder Beinen und wälzt sich ständig im Bett, bis er einschläft und damit auch ruhig ist.
\item 1W6 Erschöpfung pro angebrochenem Tag, selbst bei gelungener Unterdrückung
\item \emph{Selbstbeherrschungs}-Probe um Nebenwirkungswert erschwert zum Unterdrücken im Kampf und außerhalb (\~{}3 SR pro Probe)
\item Passende 1W20-Proben werden um den halben Nebenwirkungswert erschwert
\item Passende 3W20-Proben werden um den Nebenwirkungswert erschwert\vspace*{-\baselineskip}
\end{itemize}} & ✔ & ✔ & ✔ & ✔ \\
3 & 12 & Verlust von Lebensenergie & - & LE-Maximum sinkt um 1W6 LeP & ✔ & ✔ & ✔ & ✔ \\
3 & 13 & Ausdauerverlust/Erschöpfung/Überanstrengung & - & 
{\begin{itemize}[nosep]
\item \vspace*{-\baselineskip}AU-Maximum sinkt um 3W6 AuP
\item Bei Verlust von 12 AuP oder mehr sinkt der Erschöpfungsmaximum um 1W6\vspace*{-\baselineskip}
\end{itemize}} & ✔ & ✔ & ✔ & ✔ \\
3 & 14 & Magieresistenz sinkt & - & Magieresistenz sinkt um 1W4 Punkte & ✔ & ✔ & ✔ & ✔ \\
3 & 15 & Anfälligkeit für Krankheiten & - & Wirkung entspricht dem Nachteil \enquote{Krankheitsanfällig} & ✔ & ✔ & ✔ & ✔ \\
3 & 16 & Vergrößerung/Verkleinerung von Körperteilen & - & 
{\begin{itemize}[nosep]
\item \vspace*{-\baselineskip}betroffenes Körperteil nach Zonenkampfsystem bestimmen
\item 1W2:
\begin{enumerate*}[nosep, itemjoin={{; }}]
\item Vergrößerung auf etwa das Eineinhalbfache
\item Verkleinerung auf etwa die Hälfte
\end{enumerate*}
\item entsprechende Änderungen (FF, GE, KO, KK, GS, Kampfwerte, Gesellschaft) sind Meisterentscheid
\begin{itemize}[nosep]
\item \emph{Meisterinfo}: Eine Erhöhung gerade der Körperkraft bei einer Vergrößerung ist nicht ausgeschlossen.
\end{itemize}
\item Kleidung und Rüstung passt nicht mehr
\item Anwachsen/Schrumpfen zieht sich über einen Zeitraum von mehreren Stunden hin\vspace*{-\baselineskip}
\end{itemize}} & ✔ & ✔ & ✔ & ✔ \\
3 & 17 & Anziehend auf Feenwesen, insbesondere Kobolde & - & Der Betroffene hat die Aufmerksamkeit eines Feenwesens, häufig eines Kobolds, auf sich gezogen, der ihn entsprechend \enquote{bespaßt}. & ✔ & ✔ & ✔ & ✔ \\
3 & 18 & Verlust des Schattens & - & Der Betroffene verliert seinen Schatten. & ✔ & ✔ & ✔ & ✔ \\
3 & 19 & Sucht wird stärker & - & Die Krankheitsstufe der Sucht steigt um einen Punkt. & ✔ & ✔ & ✔ & ✔ \\
3 & 20 & Sucht wird stärker & - & Die Krankheitsstufe der Sucht steigt um einen Punkt. & ✔ & ✔ & ✔ & ✔ \\
4 & 1 & Verlust von Astralenergie & - & AE-Maximum sinkt um 1W6 AsP & ✔ & ✔ & ✔ & ✔ \\
4 & 2 & Astraler Block & - & 
{\begin{itemize}[nosep]
\item \vspace*{-\baselineskip}Die Astralregeneration des Betroffenen sinkt um eine Stufe auf der Skala meisterliche Regeneration/Regeneration II/Regeneration I/astrale Regeneration III/astrale Regeneration II/astrale Regeneration I/astraler Block.
\item Verfügt der Betroffene bereits über \enquote{astraler Block}, so wird die Regeneration um einen weiteren Punkt reduziert.\vspace*{-\baselineskip}
\end{itemize}} & ✔ & ✔ & ✔ & ✔ \\
4 & 3 & Animalische Magie & 1W5 & 
{\begin{itemize}[nosep]
\item \vspace*{-\baselineskip}Der Betroffene erhält den Nachteil \enquote{animalische Magie} in Höhe des Nebenwirkungswerts.
\item Sollte der Betroffene bereits über diesen Nachteil verfügen, so erhöht sich dessen Wert um den Nebenwirkungswert.\vspace*{-\baselineskip}
\end{itemize}} & ✔ & ✔ & ✔ & ✔ \\
4 & 3 & Festgefügtes Denken & 1W5 & 
{\begin{itemize}[nosep]
\item \vspace*{-\baselineskip}Der Betroffene erhält den Nachteil \enquote{festgefügtes Denken} in Höhe des Nebenwirkungswerts.
\item Sollte der Betroffene bereits über diesen Nachteil verfügen, so erhöht sich dessen Wert um den Nebenwirkungswert.\vspace*{-\baselineskip}
\end{itemize}} & ✔ & ✔ & ✔ & ✔ \\
4 & 4 & Feste Gewohnheit & - & 
{\begin{itemize}[nosep]
\item \vspace*{-\baselineskip}Der Betroffene erhält den Nachteil \enquote{feste Gewohnheit}. Die Ausgestaltung der durch Rauschmittelkonsum ausgelösten festen Gewohnheit obliegt dem Meister.
\item Sollte der Betroffene bereits über diesen Nachteil verfügen, so erhält er diesen ein zweites Mal mit anderer fester Gewohnheit.\vspace*{-\baselineskip}
\end{itemize}} & ✔ & ✔ & ✔ & ✔ \\
4 & 5 & Körpergebundene Kraft & - & 
{\begin{itemize}[nosep]
\item \vspace*{-\baselineskip}Der Betroffene erhält den Nachteil \enquote{körpergebundene Kraft}.
\item Dies bedeutet ebenfalls, dass die Haarfarbe sich ändert.
\item Haben die Haare bei Einsetzen der Nebenwirkung nicht die notwendige Länge, kommt der Nachteil dennoch voll zu tragen.
\item Sollte der Betroffene bereits über diesen Nachteil verfügen, so erhöht sich der Anteil der im Haar gespeicherten Kraft entsprechend, also von 1/5 auf 2/5, dann auf 3/5, auf 4/5 und letztendlich 5/5.\vspace*{-\baselineskip}
\end{itemize}} & ✔ & ✔ & ✔ & ✔ \\
4 & 6 & Lästige Mindergeister & - & 
{\begin{itemize}[nosep]
\item \vspace*{-\baselineskip}Der Betroffene erhält den Nachteil \enquote{lästige Mindergeister}.
\item Sollte der Betroffene bereits über diesen Nachteil verfügen, so bleiben Mindergeister doppelt so lang oder erscheinen bereits bei mehr als 5 AsP.\vspace*{-\baselineskip}
\end{itemize}} & ✔ & ✔ & ✔ & ✔ \\
4 & 7 & Rückschlag & - & 
{\begin{itemize}[nosep]
\item \vspace*{-\baselineskip}Der Betroffene erhält den Nachteil \enquote{Rückschlag}.
\item Verfügt der Betroffene über den Vorteil \enquote{feste Matrix} verliert er stattdessen diesen Vorteil.
\item Sollte der Betroffene bereits über diesen Nachteil verfügen, so verdoppelt sich die Stärke des Spruches.\vspace*{-\baselineskip}
\end{itemize}} & ✔ & ✔ & ✔ & ✔ \\
4 & 8 & Schwache Ausstrahlung & 1W5 & 
{\begin{itemize}[nosep]
\item \vspace*{-\baselineskip}Der Betroffene erhält den Nachteil \enquote{schwache Ausstrahlung} in Höhe des Nebenwirkungswerts.
\item Sollte der Betroffene bereits über diesen Nachteil verfügen, so erhöht sich dessen Wert um den Nebenwirkungswert.\vspace*{-\baselineskip}
\end{itemize}} & ✔ & ✔ & ✔ & ✔ \\
4 & 9 & Schwacher Astralkörper & - & 
{\begin{itemize}[nosep]
\item \vspace*{-\baselineskip}Der Betroffene erhält den Nachteil \enquote{schwacher Astralkörper}.
\item Sollte der Betroffene bereits über diesen Nachteil verfügen, so verdoppelt sich die Anzahl Würfel zur Bestimmung des zusätzlichen Verbrauchs an Astralenergie oder der Grenzwert für weitere Kosten halbiert sich.\vspace*{-\baselineskip}
\end{itemize}} & ✔ & ✔ & ✔ & ✔ \\
4 & 10 & Spruchhemmung & - & 
{\begin{itemize}[nosep]
\item \vspace*{-\baselineskip}Der Betroffene erhält den Nachteil \enquote{Spruchhemmung}.
\item Verfügt der Betroffene über den Vorteil \enquote{feste Matrix} verliert er stattdessen diesen Vorteil.
\item Sollte der Betroffene bereits über diesen Nachteil verfügen, so scheitert der Zauber, wenn ein weiterer W20 eine Zahl zeigt, die im Probenwurf bereits vorkam (außer 1 und 20). Wird diese Nebenwirkung mehrfach erwürfelt, werden für jeden weiteren W20 dennoch nur die 3W20 des ursprünglichen Probenwurfs beachtet.\vspace*{-\baselineskip}
\end{itemize}} & ✔ & ✔ & ✔ & ✔ \\
4 & 11 & Unverträglichkeit mit verarbeitetem Metall & - & 
{\begin{itemize}[nosep]
\item \vspace*{-\baselineskip}Der Betroffene erhält den Nachteil \enquote{Unverträglichkeit mit verarbeitetem Metall}.
\item Sollte der Betroffene bereits über diesen Nachteil verfügen, so erhöhen sich die Erschwernisse um \SIpct{50} (ausgehend vom einfachen Nachteil).\vspace*{-\baselineskip}
\end{itemize}} & ✔ & ✔ & ✔ & ✔ \\
4 & 12 & Wilde Magie &  - & 
{\begin{itemize}[nosep]
\item \vspace*{-\baselineskip}Der Betroffene erhält den Nachteil \enquote{wilde Magie}.
\item Verfügt der Betroffene über den Vorteil \enquote{feste Matrix} verliert er stattdessen diesen Vorteil.
\item Sollte der Betroffene bereits über diesen Nachteil verfügen, so vermindert sich die Augenzahl für einen automatischen Misserfolg um einen weiteren Punkt, also von 19 auf 18.\vspace*{-\baselineskip}
\end{itemize}} & ✔ & ✔ & ✔ & ✔ \\
4 & 13 & Zielschwierigkeiten &  - & 
{\begin{itemize}[nosep]
\item \vspace*{-\baselineskip}Der Betroffene erhält den Nachteil \enquote{Zielschwierigkeiten}.
\item Sollte der Betroffene bereits über diesen Nachteil verfügen, so erhöhen sich die Erschwernisse um \SIpct{50}.\vspace*{-\baselineskip}
\end{itemize}} & ✔ & ✔ & ✔ & ✔ \\
4 & 14 & Zögerlicher Zauberer & 1W5 & 
{\begin{itemize}[nosep]
\item \vspace*{-\baselineskip}Der Betroffene erhält den Nachteil \enquote{zögerlicher Zauberer} in Höhe des Nebenwirkungswerts.
\item Sollte der Betroffene bereits über diesen Nachteil verfügen, so erhöht sich dessen Wert um den Nebenwirkungswert.\vspace*{-\baselineskip}
\end{itemize}} & ✔ & ✔ & ✔ & ✔ \\
