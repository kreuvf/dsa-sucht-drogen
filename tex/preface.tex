% preface
\chapter*{Vorwort}\addcontentsline{toc}{chapter}{Vorwort}
\begin{figure}[b]
	\begin{center}
		\includegraphics[width=6cm]{res/Nelumbo_lutea_blossom.png}
		\caption[Weißer Lotus]{Weißer Lotus. Gemäß \citetitle{ZBA} ist eine Unterscheidung zwischen diesem und dem \emph{Gelben Lotus} nur auf alchimistischem Wege möglich.\label{img-white-lotus}}
	\end{center}
\end{figure}

Vermutlich wird sich der ein oder andere Leser schon beim Anklicken dieses Dokumentes gefragt haben, was jemanden dazu bewegt, sich überhaupt mit dem Thema \emph{Sucht \& Rauschmittel} im Rahmen eines Pen-\&-Paper-Rollenspielsystems auseinanderzusetzen. Nun, die Antwort möchte ich nicht schuldig bleiben, auch wenn sie recht simpel ist.

Ende 2018, als ich die Kampagne \enquote{Der Weiße Berg} \cite{WB1,WB2} meisterte, erhielt die Gruppe Zuwachs durch einen Zyklopäer. Im Verlaufe der Generierung zeichnete sich ab, dass der Spieler Interesse an einem Charakter mit einer \emph{Sucht} hegte. So musste ich mich als Meister also mit den Regeln zu Süchten und Rauschmitteln befassen~--~was ich aber nur oberflächlich tat, da das Abenteuer ohnehin relativ kurz vor dem Abschluss stand und es einfacher war, diese Aufgabe danach dem Meister zu überlassen, der bereits vor der Kampagne unsere Gruppe gemeistert hatte.

Ich habe die Sucht im Spiel nur handgewedelt: Nachdem die Gruppe sich halbwegs gut mit dem Sumen Eichbart gestellt hatte, übergab dieser dem Zyklopäer einige Portionen Rauschkräuter~--~streng regeltechnisch ein Verstoß gegen die Sucht nach Mibelrohr, doch wächst dieses nur an der aventurischen Westküste zwischen Salza und Mengbilla \cite[S.~251]{ZBA}. Ja, \emph{das} Salza, das zum Erzfeind Nostria gehört. Die Wahrscheinlichkeit also, dass es Mibelrohr in Andergast zu kaufen gibt, kann als eher gering betrachtet werden, zumal die unverarbeitete Pflanze ohnehin nur W6 Tage hält \cite[S.~251]{ZBA}. Dass der Charakter finanziell auch gar nicht in der Lage war sich die \SID{6} jeden zweiten Tag\footnote{Zumindest nach \citetitle{ZBA} ist das die Zeit, nach der ein Entzug einsetzt, der direkt zu \enquote{völliger Apathie} führe. Klingt ein wenig überzogen für meinen Geschmack, wenn man das mit den beim Nachteil genannten Zeiten vergleicht.} zu leisten, machte die Situation nicht besser.

Nun war der Spieler bereits rollenspielerfahren, aber ich empfand es als falsch den neuen Charakter in der Gruppe direkt mit der aus dem Regelwerk nicht direkt ersichtlichen Härte dieser Sucht zu bestrafen. Dass der Charakter eines langjährigen Mitspielers, ein, ebenfalls zyklopäischer, Magier aus Belhanka, auch noch der Bruder des Zyklopäers war, der über eine Sucht nicht allzu begeistert gewesen wäre, machte die Sache dann noch zusätzlich kompliziert. So improvisierte ich dies für die letzten drei Monate des Abenteuers und ließ den Charakter hin und wieder auch sehr einfach an Rauschmittel gelangen~--~der Konsum musste heimlich geschehen und fraß bereits genug Spielzeit.

Mit Abschluss des Abenteuers, also ab April 2019, fing ich dann an mich detaillierter mit den Suchtregeln auseinanderzusetzen und schaffte mir zuerst eine Liste aller offiziell beschriebenen Suchtmittel, dann erweiterte ich den Nachteil \emph{Sucht} so, dass auch eine Sucht nach einer bestimmten Wirkung möglich wurde. Im nächsten Schritt packte mich dann der kreative Schaffensdrang und ich erschuf nach und nach Regelmechanismen und Ereignistexte zur Beschaffung und den Nebenwirkungen dieser \enquote{zufälligen Rauschmittel}.

Als ich Mitte Mai 2019 damit fertig war, hatte der Spieler die Gruppe leider bereits (mehr oder weniger) verlassen. Damit konnte ich diese Regeln nie am Spieltisch testen und auch ein neuer Spieler, der Mitte des Jahres 2019 der Gruppe beitrat, war nicht dazu bereit seinen Thorwaler mit einer Sucht zu generieren.

Erst im März 2020 im Rahmen meiner Vorbereitungen für die zweite Staffel des Videoformats \enquote{Meisterwerkstatt~--~Das Format für vor dem Würfeln} kam ich an den Punkt, dass ich die bislang nur in einem nicht öffentlich zugänglichen Wiki\footnote{Aus Urheberrechtsgründen wird das auch nie öffentlich sein können.} lagernden Regeln in ein präsentables Format bringen wollte. Auf dem Weg zum fertigen Dokument würden alle nötigen Grafiken und Tabellen anfallen, die ich später im Video wiederverwenden könnte.

Ich hoffe, dass man nicht nur erkennt, dass viel Arbeit in dieses Dokument geflossen ist, sondern sich diese Arbeit auch gelohnt hat und beim Leser mit mehr als nur ein \enquote{er war stets bemüht} in Erinnerung bleibt.

\bigskip\noindent
{\raggedleft-- \emph{Isurandil}\par}

\vfill
\begin{center}
	\label{img-herbs-title}
	\includegraphics[width=12cm]{res/herbs-906140.png}
\end{center}
\newpage

% preface
