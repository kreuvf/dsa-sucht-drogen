\section{Beschaffung, Herstellung und Auswirkungen von Rauschmitteln}
Durch die Neuregelung des Nachteils \enquote{Sucht} wird dieser Nachteil für Spieler attraktiver. Um Meister und Spieler aber nicht im Regen stehen zu lassen, enthält dieser Abschnitt~--~der mit Abstand inhaltsreichste und aufwendigste~--~detaillierte Regeln zur Beschaffung von Rauschmitteln in der Unterwelt (\vref{beschaffung-unterwelt}), Herstellung auf eigener Faust (\vref{beschaffung-herstellung}) und den Auswirkungen insbesondere von unbenannten Rauschmitteln (\vref{auswirkungen}). \vref{fig-overview} gibt einen Überblick darüber, wie die verschiedenen Regelmechanismen ineinandergreifen.

\begin{figure}
	\begin{center}
		\includegraphics[width=16cm]{Regelübersicht_600dpi.png}
		\caption[Regelübersicht zu Beschaffung, Herstellung und Auswirkungen von Rauschmitteln]{Regelübersicht zu Beschaffung, Herstellung und Auswirkungen von Rauschmitteln. Die Sucht des Charakters erfordert den Konsum von Rauschmitteln. Diese kann der Charakter sich entweder fertig beschaffen oder versuchen geeignete Ausgangsstoffe zu finden, die dann noch verarbeitet werden müssen. Diese Wege führen in den Konsum, der mit unterschiedlichen Wirkungen und Nebenwirkungen einhergeht.\label{fig-overview}}
	\end{center}
\end{figure}

\paragraph{Designziele}
Wichtig beim Entwerfen dieser Regeln war mir, mich möglichst nah an bekannten Probenmechanismen und Talenten zu halten. Ein neues Metatalent hätte ich beispielsweise nur ungern aus der Taufe gehoben. Für die Beschaffung sind daher die Talente \emph{Gassenwissen} für den Kontakt mit der Unterwelt und \emph{Pflanzenkunde} für die Sammlung passender Ausgangsstoffe nötig. Für die Verarbeitung werden \emph{Kochen} oder \emph{Alchimie} benötigt. Einfluss auf die Wirkung hat neben der Krankheitsstufe auch der Erfolg bei Beschaffung und Verarbeitung und der \emph{Zechen}-Wert. Vorerst unbeachtet bleiben Verfallspunkte aus Rauschmittelkonsum analog zu denen aus Pakten \cite[S.~390]{WdZ} und Überlegungen zu den Kosten der Rauschmittelbeschaffung.

\subsection{Beschaffung von Rauschmitteln}
\subsubsection{Beschaffung in der Unterwelt\label{beschaffung-unterwelt}}
Für die Beschaffung in der Unterwelt ist es wichtig, zu wissen, an wen man sich wenden muss~--~und wen man lieber meidet! Ergo ist es auch eine \emph{Gassenwissen}-Probe, mit der der Charakter den Erfolg oder Misserfolg seines Unterfangen bestimmt. Die Probe wird neben gruppenüblichen Modifikatoren\footnote{In den Regeln finden sich viele Möglichkeiten, um Proben auf gesellschaftliche Talente zu modifizieren. Seien es fehlende Kulturkunden, Unterschiede im Sozialstatus \cite[S.~276]{WdH} oder Vor- und Nachteile \cite[S.~246~ff.]{WdH} des Charakters. Ein arroganter (9) horasischer Magier aus der Mittelschicht (Sozialstatus 8), der nach einem Unfall entstellt ist (\emph{unansehnlich}), wird bei einem durchschnittlichen Albenhuser Zwerg selbst bei einfachsten Bitten Probleme bekommen, da die Probe um über +12 erschwert ist. Da die Regelanwendung aber glücklicherweise nicht von einer Regelpolizei überwacht und durchgesetzt wird, wird dies in der jeder Gruppe ein wenig anders aussehen, weshalb ich darauf verzichte jeden Modifikator einzeln aufzuführen.} um die \emph{halbe Krankheitsstufe der Sucht} erschwert. Zum Verhindern von Metagaming empfehle ich, dass der Meister diese Probe verdeckt würfelt.

Spieler und Meister sollten vor der Probe festlegen, wie offen der Charakter erkennbar sein möchte, wie viel Geld er dabei hat und wie viele Portionen des Rauschmittels er sucht. Keiner dieser Punkte ist für die Probe wichtig, aber möglicherweise beim Scheitern von Belang.

Das Ergebnis des Wurfs wird wie folgt interpretiert:
\paragraph{Doppel-1}
Ein Schmuggler sitzt gerade auf genau der gesuchten Ware, die er aber dringend loswerden muss. Der Charakter erhält saubere Ware zu einem Fünftel des Normalpreises. Die Qualität liegt bei den vollen TaP* + 4W3.

\paragraph{Gelungen}
Der Charakter wird an einen entsprechenden \enquote{Händler} verwiesen. Die TaP* werden später zur Beurteilung der Qualität der Ware herangezogen.

\paragraph{Misslungen}
1W20-Wurf gegen den Betrag der TaP*.\footnote{Beispiel: Wenn aus der Probe −10 TaP* kommen, ist der Betrag davon 10 TaP* und es muss mit 1W20 gegen einen Wert von 10 gewürfelt werden. Wie bei DSA üblich, ist die Probe bestanden, wenn das Würfelergebnis kleiner oder gleich dem Wert, hier: 10, ist. Mit einer 5 ist der Wurf also gelungen, mit einer 17 misslungen. Je stärker der Wurf also danebengegangen ist, je weniger TaP* also angesammelt wurden, desto höher ist die Wahrscheinlichkeit, dass der Wurf gelingt und es zu ernsten Konsequenzen kommen kann. Eine letzte, mathematische, Anmerkung zum Verständnis: −10 TaP* sind \emph{weniger} als −3 TaP*.}
\begin{itemize}
	\item Gelungen: Es wird mit einem explodierenden W20\footnote{Explodierende Würfe kommen in DSA4.1 sehr selten vor, weshalb ich das Prinzip hier kurz erläutern möchte: Ein explodierender W20 wird wie ein gewöhnlicher W20 gewürfelt. Zeigt der Würfel aber eine 20, muss erneut gewürfelt und das Ergebnis des ersten Wurfs zum zweiten addiert werden. Zeigt auch der zweite Wurf wieder eine 20, so muss abermals gewürfelt und addiert werden. Theoretisch sind damit Ergebnisse in \emph{unendlicher} Höhe möglich.} gewürfelt und das Ergebnis zum Betrag der TaP* aus der Gassenwissenprobe addiert. Um die hinter diesem Wert stehende Auswirkung zu erfahren, diesen in \vref{tbl-auswirkungen} nachschlagen.
	\item Misslungen: Der Charakter findet niemanden, der ihm die Wunschdrogen verkauft, ohne dass es weitere Konsequenzen nach sich zieht.
\end{itemize}

\paragraph{Doppel-20}
Wie misslungene Probe, der Prüfwurf gegen den Betrag der TaP* gilt aber automatisch als gelungen und es werden zusätzlich 5 zum Ergebnis zur Bestimmung der genauen Auswirkungen addiert.

% Use \vspace*{-\baselineskip} to circumvent strut after itemize giving a blank line
% Source: https://tex.stackexchange.com/a/141674/92368
{
\rowcolors{2}{tablelightblue}{white}
\begin{tabularx}{\linewidth}{p{1.2cm}p{14.8cm}}
	\caption[Auswirkungen misslungener Rauschmittelbeschaffung in der Unterwelt]{Auswirkungen misslungener Rauschmittelbeschaffung in der Unterwelt. Misslungene Gassenwissen-Proben zur Beschaffung von Rauschmitteln in der Unterwelt können bestimmte Auswirkungen haben, die über den unter \vref{beschaffung-unterwelt} beschriebenen Mechanismus ein Ergebnis von mindestens 2 liefern. Der Schweregrad der Auswirkung steigt mehr oder weniger mit dem Ergebnis an. Da etliche Auswirkungen auch zu Schadenspunkten führen, sei dem Meister ans Herz gelegt, dass diese Tabelle \emph{nicht} das Ziel verfolgt, dem Süchtigen möglichst den Spielspaß zu vermiesen und ihn ins Grab zu führen. Kein Charakter soll durch die bloße Benutzung dieser Auswirkungentabelle befürchten müssen zu sterben. Sollte eine bestimmte Auswirkung unpassend erscheinen, liegt es am Meister sich in der Nähe des Ergebnisses umzuschauen, um eine geeignetere Auswirkung zu finden.\label{tbl-auswirkungen}} \\
	\toprule
	{\cellcolor{white}Ergebnis} & {\cellcolor{white}Auswirkung} \\
	\hline \endfirsthead
	\caption[]{\cellcolor{white}\textit{Fortsetzung von der vorhergehenden Seite}} \\
	\toprule
	{\cellcolor{white}Ergebnis} & {\cellcolor{white}Auswirkung} \\
	\hline
	\endhead
	\hline
	\multicolumn{2}{r}{\cellcolor{white}\textit{Fortsetzung auf der nächsten Seite}} \\
	\endfoot
	\bottomrule
	\endlastfoot
	\TablesafeInputIfFileExists{res/consequences/consequences.tex}{}{}
\end{tabularx}
}
