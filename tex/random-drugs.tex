\section{Beschaffung, Herstellung und Auswirkungen von Rauschmitteln}
Durch die Neuregelung des Nachteils \enquote{Sucht} wird dieser Nachteil für Spieler attraktiver. Um Meister und Spieler aber nicht im Regen stehen zu lassen, enthält dieser Abschnitt~--~der mit Abstand inhaltsreichste und aufwendigste~--~detaillierte Regeln zur Beschaffung von Rauschmitteln in der Unterwelt (\vref{beschaffung-unterwelt}), Herstellung auf eigener Faust (\vref{beschaffung-herstellung}) und den Auswirkungen insbesondere von unbenannten Rauschmitteln (\vref{auswirkungen}). \vref{fig-overview} gibt einen Überblick darüber, wie die verschiedenen Regelmechanismen ineinandergreifen.

\begin{figure}
	\begin{center}
		\includegraphics[width=16cm]{Regelübersicht_600dpi.png}
		\caption[Regelübersicht zu Beschaffung, Herstellung und Auswirkungen von Rauschmitteln]{Regelübersicht zu Beschaffung, Herstellung und Auswirkungen von Rauschmitteln. Die Sucht des Charakters erfordert den Konsum von Rauschmitteln. Diese kann der Charakter sich entweder fertig beschaffen oder versuchen geeignete Ausgangsstoffe zu finden, die dann noch verarbeitet werden müssen. Diese Wege führen in den Konsum, der mit unterschiedlichen Wirkungen und Nebenwirkungen einhergeht.\label{fig-overview}}
	\end{center}
\end{figure}

\paragraph{Designziele}
Wichtig beim Entwerfen dieser Regeln war mir, mich möglichst nah an bekannten Probenmechanismen und Talenten zu halten. Ein neues Metatalent hätte ich beispielsweise nur ungern aus der Taufe gehoben. Für die Beschaffung sind daher die Talente \emph{Gassenwissen} für den Kontakt mit der Unterwelt und \emph{Pflanzenkunde} für die Sammlung passender Ausgangsstoffe nötig. Für die Verarbeitung werden \emph{Kochen} oder \emph{Alchimie} benötigt. Einfluss auf die Wirkung hat neben der Krankheitsstufe auch der Erfolg bei Beschaffung und Verarbeitung und der \emph{Zechen}-Wert. Vorerst unbeachtet bleiben Verfallspunkte aus Rauschmittelkonsum analog zu denen aus Pakten \cite[S.~390]{WdZ} und Überlegungen zu den Kosten der Rauschmittelbeschaffung.

\subsection{Beschaffung von Rauschmitteln}
\subsubsection{Beschaffung in der Unterwelt\label{beschaffung-unterwelt}}
Für die Beschaffung in der Unterwelt ist es wichtig, zu wissen, an wen man sich wenden muss~--~und wen man lieber meidet! Ergo ist es auch eine \emph{Gassenwissen}-Probe, mit der der Charakter den Erfolg oder Misserfolg seines Unterfangens bestimmt. Die Probe wird neben gruppenüblichen Modifikatoren\footnote{In den Regeln finden sich viele Möglichkeiten, um Proben auf gesellschaftliche Talente zu modifizieren. Seien es fehlende Kulturkunden, Unterschiede im Sozialstatus \cite[S.~276]{WdH} oder Vor- und Nachteile \cite[S.~246~ff.]{WdH} des Charakters. Ein arroganter (9) horasischer Magier aus der Mittelschicht (Sozialstatus 8), der nach einem Unfall entstellt ist (\emph{unansehnlich}), wird bei einem durchschnittlichen Albenhuser Zwerg selbst bei einfachsten Bitten Probleme bekommen, da die Probe um über +12 erschwert ist. Da die Regelanwendung aber glücklicherweise nicht von einer Regelpolizei überwacht und durchgesetzt wird, wird dies in der jeder Gruppe ein wenig anders aussehen, weshalb ich darauf verzichte jeden Modifikator einzeln aufzuführen.} um die \emph{halbe Krankheitsstufe der Sucht} erschwert. Zum Verhindern von Metagaming empfehle ich, dass der Meister diese Probe verdeckt würfelt.

Spieler und Meister sollten vor der Probe festlegen, wie offen der Charakter erkennbar sein möchte, wie viel Geld er dabei hat und wie viele Portionen des Rauschmittels er sucht. Keiner dieser Punkte ist für die Probe wichtig, aber möglicherweise beim Scheitern von Belang.

Das Ergebnis des Wurfs wird wie folgt interpretiert:
\begin{itemize}
	\item Doppel-1: Ein Schmuggler sitzt gerade auf genau der gesuchten Ware, die er aber dringend loswerden muss. Der Charakter erhält saubere Ware zu einem Fünftel des Normalpreises. Die Qualität liegt bei den vollen TaP* + 4W3.
	\item Gelungen: Der Charakter wird an einen entsprechenden \enquote{Händler} verwiesen. Die TaP* werden später zur Beurteilung der Qualität der Ware herangezogen.
	\item Misslungen: 1W20-Wurf gegen den Betrag der TaP*.\footnote{Beispiel: Wenn aus der Probe −10 TaP* kommen, ist der Betrag davon 10 TaP* und es muss mit 1W20 gegen einen Wert von 10 gewürfelt werden. Wie bei DSA üblich, ist die Probe bestanden, wenn das Würfelergebnis kleiner oder gleich dem Wert, hier: 10, ist. Mit einer 5 ist der Wurf also gelungen, mit einer 17 misslungen. Je stärker der \emph{Gassenwissen}-Wurf also danebengegangen ist, je weniger TaP* also angesammelt wurden, desto höher ist die Wahrscheinlichkeit, dass der Wurf gelingt und es zu ernsten Konsequenzen kommen kann. Eine letzte, mathematische, Anmerkung zum Verständnis: −10 TaP* sind \emph{weniger} als −3 TaP*.}
		\begin{itemize}
			\item Gelungen: Es wird mit einem explodierenden W20\footnote{Explodierende Würfe kommen in DSA4.1 sehr selten vor, weshalb ich das Prinzip hier kurz erläutern möchte: Ein explodierender W20 wird wie ein gewöhnlicher W20 gewürfelt. Zeigt der Würfel aber eine 20, muss erneut gewürfelt und das Ergebnis des ersten Wurfs zum zweiten addiert werden. Zeigt auch der zweite Wurf wieder eine 20, so muss abermals gewürfelt und addiert werden. Theoretisch sind damit Ergebnisse in \emph{unendlicher} Höhe möglich.} gewürfelt und das Ergebnis zum Betrag der TaP* aus der Gassenwissenprobe addiert. Um die hinter diesem Wert stehende Auswirkung zu erfahren, diesen in \vref{tbl-auswirkungen} nachschlagen.
			\item Misslungen: Der Charakter findet niemanden, der ihm die Wunschdrogen verkauft, ohne dass es weitere Konsequenzen nach sich zieht.
		\end{itemize}
	\item Doppel-20: Wie misslungene Probe, der Prüfwurf gegen den Betrag der TaP* gilt aber automatisch als gelungen und es werden zusätzlich 5 zum Ergebnis zur Bestimmung der genauen Auswirkungen addiert.
\end{itemize}

% Use \vspace*{-\baselineskip} to circumvent strut after itemize giving a blank line
% Source: https://tex.stackexchange.com/a/141674/92368
{
\rowcolors{2}{tablelightblue}{white}
\begin{tabularx}{\linewidth}{p{1.2cm}p{14.8cm}}
	\caption[Auswirkungen misslungener Rauschmittelbeschaffung in der Unterwelt]{Auswirkungen misslungener Rauschmittelbeschaffung in der Unterwelt. Misslungene Gassenwissenproben zur Beschaffung von Rauschmitteln in der Unterwelt können bestimmte Auswirkungen haben, die über den unter \vref{beschaffung-unterwelt} beschriebenen Mechanismus ein Ergebnis von mindestens 2 liefern. Der Schweregrad der Auswirkung steigt mehr oder weniger mit dem Ergebnis an. Da etliche Auswirkungen auch zu Schadenspunkten führen, sei dem Meister ans Herz gelegt, dass diese Tabelle \emph{nicht} das Ziel verfolgt, dem Süchtigen möglichst den Spielspaß zu vermiesen und ihn ins Grab zu führen. Kein Charakter soll durch die bloße Benutzung dieser Auswirkungentabelle befürchten müssen zu sterben. Sollte eine bestimmte Auswirkung unpassend erscheinen, liegt es am Meister sich in der Nähe des Ergebnisses umzuschauen, um eine geeignetere Auswirkung zu finden.\label{tbl-auswirkungen}} \\
	\toprule
	{\cellcolor{white}Ergebnis} & {\cellcolor{white}Auswirkung} \\
	\hline \endfirsthead
	\caption[]{\cellcolor{white}\textit{Fortsetzung von der vorhergehenden Seite}} \\
	\toprule
	{\cellcolor{white}Ergebnis} & {\cellcolor{white}Auswirkung} \\
	\hline
	\endhead
	\hline
	\multicolumn{2}{r}{\cellcolor{white}\textit{Fortsetzung auf der nächsten Seite}} \\
	\endfoot
	\bottomrule
	\endlastfoot
	\TablesafeInputIfFileExists{res/consequences/consequences.tex}{}{}
\end{tabularx}
}

\subsubsection{Herstellung auf eigene Faust\label{beschaffung-herstellung}}
Mit dem passenden Wissen ausgestattet lassen sich Rauschmittel auch selbst herstellen. Die Gefahr dabei entdeckt zu werden, ist deutlich geringer als bei der Beschaffung in der Unterwelt. Dafür werden aber auch bis zu zwei Proben nötig: \emph{Kräutersuche}\footnote{Ich habe mich bewusst auf Pflanzen beschränkt. Als Anregung für Spieler und Meister sei aber erwähnt, dass neben pflanzlichen Inhaltsstoffen auch tierische oder mineralische vorstellbar sind, von Magie in Form von Humuselementaren oder einem \emph{Haselbusch und Ginsterkraut} ganz zu schweigen.} für die Beschaffung der Ausgangsstoffe und entweder \emph{Alchimie} oder \emph{Kochen} in einem wenigstens archaischen Labor.

\paragraph{Kräutersuche}
Für die \emph{Kräutersuche} zählt neben allen anderen Modifikatoren \cite[S.~223/224]{ZBA}\cite[S.~113]{WdE} die \emph{halbe Krankheitsstufe der Sucht} als Erschwernis. Die Verdopplung der Suchdauer und damit die Veranderthalbfachung des Talentwerts sind wie üblich zulässig.

Das Ergebnis des Wurfs wird wie folgt interpretiert:
\begin{itemize}
	\item Doppel-1: Der Charakter findet eine Menge von 1~+~1W3~+~1 pro halbe Krankheitsstufe der Sucht TaP* Portionen mit einer Qualität von TaP*~+~4W3.
	\item Gelungen: Der Charakter findet eine Menge von 1~+~1 pro halbe Krankheitsstufe der Sucht TaP* Portionen mit einer Qualität von TaP*.
	\item Misslungen: Der Charakter findet keine geeigneten Kräuter. In diesem Falle ist keine Herstellprobe möglich.
	\item Doppel-20: Der Charakter findet eine Menge von 1~+~1 pro halbe Krankheitsstufe der Sucht des Betrags der TaP* Portionen mit einer Qualität von −|TaP*|~−~4W3.
\end{itemize}

\paragraph{Herstellung der Rauschmittel}
Im nächsten Schritt geht es an die Herstellung der Rauschmittel. Dabei kann der Charakter wählen, ob er das Talent \emph{Alchimie} benutzen will oder das Talent \emph{Kochen}. Im ersten Fall darf die Talentspezialisierung \emph{Rauschmittel}, im zweiten Fall \emph{Tränke} verwendet werden. Es wird in beiden Fällen mindestens ein archaisches Labor benötigt~--~wer also das Glück hat Zugriff auf ein besseres Labor zu haben, kann auch von den entsprechenden Erleichterungen profitieren \cite[S.~16]{WdA}. Die Probe wird erschwert um die \emph{halbe Krankheitsstufe der Sucht} und modifiziert um die Qualität der Ausgangsmaterialien, bei positiver Qualität also erleichtert und bei negativer Qualität erschwert.

Das Ergebnis des Wurfs wird wie folgt interpretiert:
\begin{itemize}
	\item Doppel-1: Der Charakter setzt die Ausgangsstoffe optimal um, sodass für jede Portion eine Anwendung entsteht. Die Qualität beträgt TaP*~+~4W3.
	\item Gelungen: Der Charakter erhält eine Menge von 1~+~1 pro halbe Krankheitsstufe der Sucht TaP* Anwendungen, aber nicht mehr als die Anzahl Portionen der Ausgangsstoffe. Die Qualität beträgt TaP*.
	\item Misslungen: Dem Charakter misslingt die Zubereitung, sämtliche Ausgangsstoffe wurden dabei verdorben. Immerhin ist es noch rechtzeitig aufgefallen.
	\item Doppel-20: Der Charakter erhält eine Menge von 1~+~1 pro halbe Krankheitsstufe der Sucht des Betrags der TaP* Anwendungen, auch über die Anzahl Portionen der Ausgangsstoffe hinaus. Die Qualität beträgt −|TaP*|~−~4W3.
\end{itemize}

\subsection{Auswirkungen des Rauschmittelkonsums\label{auswirkungen}}

\begin{table}
	\centering
	\caption[Rauschmittelkonsum: Interpretation des Ergebnisses der Zechenprobe]{Interpretation des Ergebnisses der Zechenprobe für den Rauschmittelkonsum. Über die erste Spalte werden die Auswirkung des Konsums ermittelt. In der Regel tritt die gewünschte Wirkung ein und ein Prüfwurf wird nötig, um zu entscheiden, wie gut das Rauschmittel war. Gelingt diese Probe, so ergeben sich die Anzahlen der kurzfristigen, mittelfristigen, langfristigen und permanenten Nebenwirkungen aus der vorletzten Spalte; misslingt sie, so aus der letzten Spalte.\label{tbl-auswirkungen-konsum}}
	\begin{tabular}{lllcc}
		\toprule
		 &  &  & \multicolumn{2}{c}{Anzahl Nebenwirkungen} \\
		\cmidrule(lr){4-5}
		\multirow{-2}*{\raisebox{0.35em}{Ergebnis}} & \multirow{-2}*{\raisebox{0.35em}{Hauptwirkung?}} & \multirow{-2}*{\raisebox{0.35em}{Prüfwurf}} & {Gelungen} & {Misslungen} \\
		\hline
		Doppel-1 &  & {keiner} & 0/0/0/0 & 0/0/0/0 \\
		\cmidrule{3-5}
		≥ 0 &  & 1W20 gegen Ergebnis & 0/0/0/0 & 1/0/0/0 \\
		\cmidrule{3-5}
		−1 bis −5 &  &  & 1/0/0/0 & 1/1/0/0 \\
		\arrayrulecolor{white}\cmidrule{4-5}\arrayrulecolor{digibluedark}
		−6 bis −10 &  & \multirow{-2}*{\raisebox{0.15em}{1W20 gegen |TaP*|}} & 1/1/0/0 & 1/1/1/0 \\
		\cmidrule{3-5}
		−11 bis −15 &  &  & 1/1/1/0 & 2/1/1/0 \\
		\arrayrulecolor{white}\cmidrule{4-5}\arrayrulecolor{digibluedark}
		−16 bis −20 & \multirow{-6}*{\raisebox{2em}{tritt ein}} &  & 2/1/1/0 & 2/2/1/0 \\
		\cmidrule{2-2}
		−21 bis −25 & tritt abgeschwächt ein &  & 2/2/1/0 & 1/2/2/0 \\
		\cmidrule{2-2}
		−26 bis −30 &  &  & 1/2/2/0 & 1/1/3/0 \\
		\arrayrulecolor{white}\cmidrule{4-5}\arrayrulecolor{digibluedark}
		< −30 &  & \multirow{-5}*{\raisebox{1.7em}{1W20 gegen 10}} & 1/1/3/0 & 1/1/2/1 \\
		\cmidrule{3-5}
		Doppel-20 & \multirow{-3}*{\raisebox{0.8em}{tritt nicht ein}} & 1W20 gegen 19 & 1/1/2/1 & 1/1/1/2 \\
		\bottomrule
	\end{tabular}
\end{table}

{
\rowcolors{2}{tablelightblue}{white}
\begin{table}
	\centering
	\caption[Nebenwirkungsklassen]{Nebenwirkungsklassen mit Wirkbeginn und -ende.\label{tbl-nebenwirkungsklassen}}
	\begin{tabularx}{\textwidth}{p{3.25cm}p{6.75cm}p{5.7cm}}
		\toprule
		Nebenwirkungsklasse & Wirkbeginn & Wirkende \\
		\hline
		kurzfristig & während des Konsums & 3W6 h nach Ende der Hauptwirkung \\
		mittelfristig & 1W3 Tage nach Ende der Hauptwirkung & 3~+~1W6 Tage \\
		langfristig & 1W3 Wochen nach Ende der Hauptwirkung & 3~+~1W6 Wochen \\
		permanent & während des Konsums & nur nach gelungener Probe auf \emph{Heilkunde Gifte/Krankheiten/Seele} \\
		\bottomrule
	\end{tabularx}
\end{table}
}
Am Ende beider Beschaffungswege stehen Rauschmittel. Diesen ist eine bestimmte Qualität zugeordnet, die hauptsächlich darstellt, wie alchimistisch rein das Rauschmittel ist. Um zu überprüfen, welche Auswirkungen der Konsum hat, wird auf das Talent \emph{Zechen} geprobt. Die Probe wird um die Qualität der Rauschmittel erleichtert, bei negativen Qualitäten also erschwert. Als Erschwernis kommt die \emph{halbe Krankheitsstufe der Sucht} hinzu, eine passende Spezialisierung auf \emph{Zechen} kann aber verwendet werden.

Das Ergebnis kann dann aus \vref{tbl-auswirkungen-konsum} abgelesen werden. Mit etwas Glück und Können kommt der Charakter ganz um Nebenwirkungen herum. Ob es zu Nebenwirkungen kommt, wird über einen Prüfwurf entschieden. Wie dieser aussieht, hängt davon ab, wie gut das Ergebnis ist. Grundsätzlich gilt aber: Je schlechter das Ergebnis, desto mehr Nebenwirkungen gibt es und umso länger halten diese an. Sollte der Charakter gar großes Pech haben, so kann es sein, dass die Dosis nicht ausgereicht hat und eine weitere nötig wird~--~die eine weitere Probe mit der erneuten Gefahr sich weitere Nebenwirkungen einzufangen bedeutet.

Die Nebenwirkungen werden ausgewürfelt: Zuerst werden für jede Nebenwirkung Wirkbeginn und -ende ausgewürfelt (siehe \vref{tbl-nebenwirkungsklassen}), dann die Nebenwirkungen selbst ausgewürfelt. Für magiebegabte Charaktere wird mit 1W4 und 1W20, für alle anderen Charaktere mit 1W3 und 1W20 gewürfelt. Die Ergebnisse aus beiden Würfen können in den ersten beiden Spalten der Tabelle \vref{tbl-nebenwirkungen} nachgeschlagen werden. Dort müssen mitunter weitere Würfe getätigt werden, um die Auswirkungen genauer zu bestimmen. Sind mehrere Werte von einer ausgewürfelten Veränderung betroffen, so wird grundsätzlich für jeden Wert einzeln gewürfelt. Daneben ist auch erkennbar, ob die Nebenwirkung für die vorgesehene Nebenwirkungsklasse empfohlen wird.

Wird mehrmals dieselbe Nebenwirkung erwürfelt, entscheidet der Meister, ob die Nebenwirkung doppelt so lange anhält, doppelt so stark ausfällt (falls möglich) oder stattdessen eine der angrenzenden Nebenwirkungen gewählt wird. Daher sind die Nebenwirkungen auch grob nach Ähnlichkeit sortiert. Ebenso entscheidet der Meister, ob Nebenwirkungen, die für die Nebenwirkungsklasse nicht empfohlen werden, erneut gewürfelt werden. Führt der W20-Wurf zu keiner Nebenwirkung, da für dieses Ergebnis noch keine Nebenwirkung definiert ist, wird der W20-Wurf wiederholt. Ändern sich Eigenschaften, so zieht dies grundsätzlich keine Neuberechnung abgeleiteter Werte nach sich. Negative Regenerationswerte zählen wie 0.

