\section{Überarbeitete Regeln zu Sucht und Rauschmitteln}
Der Nachteil \emph{Sucht} muss flexibler werden. Dazu wird es nötig neben benannten Rauschmitteln wie \emph{Alkohol}, \emph{Mibelrohr} oder \emph{Samthauch} allgemeine Wirkungen zuzulassen. Daneben erhalten die Bereiche \emph{Beschaffung}, \emph{Herstellung} und \emph{Nebenwirkungen} ausführliche Regelmechanismen, um innerhalb von rund fünf Minuten und vielleicht zehn Würfen den Nachteil am Spieltisch abhandeln zu können.

\subsection{Sucht (Nachteil)}
Die Flexibilität des Nachteils versuche ich auf zwei Wegen zu steigern:
\begin{enumerate}
	\item Ersetzung der starren Krankheitsstufe für \emph{benannte Suchtmittel}\footnote{Unter \enquote{benannte Suchtmittel} fallen all jene Suchtmittel, die einen eigenen Namen haben wie beispielsweise \emph{Alkohol}, \emph{Mibelrohr} oder \emph{Samthauch}. Dieser Begriff dient der Abgrenzung zu den \emph{unbenannten Suchtmitteln}.} durch einen Bereich.\footnote{Die Sonderbehandlung für Alkohol entfällt.} Der Meister hat das letzte Wort, kann sich aber an der überarbeiteten \hyperref[tbl-suchtmittel-revised]{Tabelle der Suchtmittel} (S.~\pageref{tbl-suchtmittel-revised}) orientieren, die empfohlene Bereiche für die Krankheitsstufen der einzelnen Suchtmittel enthält.
	\item \emph{Unbenannte Suchtmittel}\footnote{Im Gegensatz zu benannten Suchtmitteln haben \enquote{unbenannte Suchtmittel} keinen vordefinierten Namen. Sie dienen lediglich als Platzhalter für die mit ihnen verbundene Wirkung.} mit allgemeinen Wirkungen. Die Sucht bezieht sich auf eine allgemeine Wirkung, die Spieler und Meister vorab festlegen müssen; dabei sollte man sich an den \hyperref[tbl-suchtmittel]{bereits beschriebenen Suchtmitteln} (S.~\pageref{tbl-suchtmittel}) orientieren. Durch die Festlegung auf eine Wirkung statt eines Suchtmittels kann der Charakter sich am lokal verfügbaren Angebot orientieren, was auch einem reisenden Charakter erlaubt diesen Nachteil zu wählen. Die Krankheitsstufe kann im Rahmen von 1 bis 30 gewählt werden.
\end{enumerate}

\begin{table}
	\centering
	\caption[Suchtmittel mit flexiblen Krankheitsstufen]{Überarbeitete Übersicht aller Suchtmittel aus \citetitle{WdH}, \citetitle{WdA} und \citetitle{ZBA} mit empfohlenen Bereichen für die Krankheitsstufe statt fester Werte. Die offiziellen Angaben finden sich in \vref{tbl-suchtmittel}.\label{tbl-suchtmittel-revised}}
	\rowcolors{2}{tablelightblue}{white}
	\begin{tabular}{lc}
		\toprule
		{Name des Suchtmittels} & {Bereiche der Krankheitsstufe der Sucht} \\
		\hline
		Alkohol & 2~--~8 \\
		Boronwein & 6~--~12 \\
		Feuerschlick (Ilaris-Sucht) & 6~--~12 \\
		Libellengras & 1~--~8 \\
		Malomis & 10~--~20 \\
		Marbos Odem & 4~--~10 \\
		Mibelrohr & 4~--~10 \\
		Moarana & 2~--~8 \\
		Purpurmohn & 13 \\
		Regenbogenstaub & 6~--~12 \\
		Samthauch & 1~--~8 \\
		Schwarzer Wein & 5~--~14 \\
		Vierblättrige Einbeere & 5~--~14 \\
		Wachtrunk & 2~--~8 \\
		Wasserrausch & 1~--~8 \\
		Willenstrunk & 5~--~14 \\
		\bottomrule
	\end{tabular}
\end{table}

\paragraph{Entzug und Konsumhäufigkeit}
Die Auswirkungen des Entzugs verhalten sich wie in \citetitle{WdH} und \citetitle{WdS} beschrieben. Allerdings treten Entzugssymptome nicht nach einer festen Zeitspanne seit dem letzten Konsum auf, sondern werden per Würfelwurf durch den Meister verdeckt ermittelt. Die Krankheitsstufe der Sucht gibt dabei die \emph{Konsumhäufigkeit} innerhalb von dreißig Tagen an. Teilt man die Konsumhäufigkeit durch die Krankheitsstufe der Sucht, so erhält man die durchschnittliche Konsumpause. Für ausgewählte Werte hält \vref{tbl-konsum} Konsumhäufigkeiten und die sich daraus ergebenden durchschnittlichen Konsumpausen, meine Empfehlungen für passende Würfelproben und zum Vergleich die Statistik des Wurfs bereit. Für jede Stufe des Entzugs wird dabei ein eigener Wurf fällig. Das heißt, dass nach Ablauf der Zeit bis zum nächsten Konsum der Entzug auf Stufe 1 beginnt und der Meister wieder durch Würfelwurf verdeckt ermittelt, wann die zweite Stufe des Entzugs beginnt.

\begin{table}
	\centering
	\caption[Auswirkung der Krankheitsstufe auf Entzug und Konsumhäufigkeit]{Auswirkung der Krankheitsstufe auf Entzug und Konsumhäufigkeit. Neben der sich aus der Krankheitsstufe und damit auch der Konsumhäufigkeit innerhalb von dreißig Tagen ergebenden durchschnittlichen Konsumpause sind Empfehlungen für einen Wurf auf die Dauer bis zum Einsetzen des Entzugs und die dazugehörigen minimalen, durchschnittlichen und maximalen Würfelergebnisse zum Vergleich angegeben.\label{tbl-konsum}}
	\rowcolors{2}{tablelightblue}{white}
	\begin{threeparttable}
		\begin{tabular}{cccccc}
			\toprule
			 & \multicolumn{2}{c}{Konsumpause in d} & \multicolumn{3}{c}{Wurfstatistik} \\
			\cmidrule(lr){2-3}
			\cmidrule(lr){4-6}
			\multirow{-2}*{Konsumhäufigkeit pro \SId{30}} & {Durchschnitt} & {Wurf} & {min.} & {Durchschnitt} & {max.} \\
			\hline
			2 & 15 & 4 + 1W20 & 5 & 14,5 & 24 \\
			5 & 6,0 & 1 + 1W10 & 2 & 6,5 & 11 \\
			8 & 3,75 & 1W6 & 1 & 3,5 & 6 \\
			12 & 2,5 & 1W4 & 1 & 2,5 & 4 \\
			15 & 2,0 & 1W3 & 1 & 2,0 & 3 \\
			20 & 1,5 & 1W4 − 1 & 0\tnotex{tnote:null} & 1,5 & 3 \\
			30 & 1,0 & 1W3 − 1 & 0\tnotex{tnote:null} & 1,0 & 2 \\
			\bottomrule
		\end{tabular}
		\begin{tablenotes}
			\item\label{tnote:null} Jedes Mal, wenn als Ergebnis eine Null erzielt wird, bedeutet dies, dass umgehend eine weitere Dosis und ein weiterer Wurf nötig werden. Es ist Meisterentscheid, ob dabei die Sucht so stark ist, dass nach Ende des aktuellen Rausches sofort ein weiterer Rausch gebraucht wird, oder noch kein Rausch eingetreten ist. Die Auswirkungen \fxnote{Verweis auf Auswirkungen bei unbenannten Suchtmitteln hinzufügen.} gelten trotzdem für \emph{jede} konsumierte Dosis. Meistern empfehle ich daher Spieler deutlich auf die Gefahr hinzuweisen, sich für eine Sucht mit derart hohen Werten zu entscheiden.
		\end{tablenotes}
	\end{threeparttable}
\end{table}

\paragraph{Generierungspunkte}
Um die Einschränkungen benannter Rauschmittel zu berücksichtigen, gibt es für diese auch weiterhin GP in Höhe der doppelten Krankheitsstufe. Wer sich hingegen für allgemeine Wirkungen entscheidet, erhält nur GP in der Höhe der einfachen Krankheitsstufe.
