\section{Überarbeitete Regeln zu Sucht und Rauschmitteln}
Der Nachteil \enquote{Sucht} muss flexibler werden. Dazu wird es nötig neben benannten Rauschmitteln wie \enquote{Alkohol}, \enquote{Mibelrohr} oder \enquote{Samthauch} allgemeine Wirkungen zuzulassen. Daneben erhalten die Bereiche \enquote{Beschaffung}, \enquote{Herstellung} und \enquote{Nebenwirkungen} ausführliche Regelmechanismen, um innerhalb von rund fünf Minuten und vielleicht zehn Würfen den Nachteil am Spieltisch abhandeln zu können.

\subsection{Sucht (Nachteil)}
Die Flexibilität des Nachteils versuche ich auf zwei Wegen zu steigern:
\begin{enumerate}
	\item Ersetzung der starren Krankheitsstufe für \emph{benannte Suchtmittel}\footnote{Unter \enquote{benannte Suchtmittel} fallen all jene Suchtmittel, die einen eigenen Namen haben wie beispielsweise \enquote{Alkohol}, \enquote{Mibelrohr} oder \enquote{Samthauch}. Dieser Begriff dient der Abgrenzung zu den \emph{unbenannten Suchtmitteln}.} durch einen Bereich.\footnote{Die Sonderbehandlung für Alkohol entfällt.} Der Meister hat das letzte Wort, kann sich aber an der überarbeiteten \hyperref[tbl-suchtmittel-revised]{Tabelle der Suchtmittel} (S.~\pageref{tbl-suchtmittel-revised}) orientieren, die empfohlene Bereiche für die Krankheitsstufen der einzelnen Suchtmittel enthält.
	\item \emph{Unbenannte Suchtmittel}\footnote{Im Gegensatz zu benannten Suchtmitteln haben \enquote{unbenannte Suchtmittel} keinen vordefinierten Namen. Sie dienen lediglich als Platzhalter für die mit ihnen verbundene Wirkung.} mit allgemeinen Wirkungen. Die Sucht bezieht sich auf eine allgemeine Wirkung, die Spieler und Meister vorab festlegen müssen; dabei sollte man sich an den \hyperref[tbl-suchtmittel]{bereits beschriebenen Suchtmitteln} (S.~\pageref{tbl-suchtmittel}) orientieren. Durch die Festlegung auf eine Wirkung statt eines Suchtmittels kann der Charakter sich am lokal verfügbaren Angebot orientieren, was auch einem reisenden Charakter erlaubt diesen Nachteil zu wählen. Die Krankheitsstufe kann im Rahmen von 1 bis 30 gewählt werden.
\end{enumerate}

\begin{table}
	\centering
	\caption[Suchtmittel mit flexiblen Krankheitsstufen]{Überarbeitete Übersicht aller Suchtmittel aus \citetitle{WdH}, \citetitle{WdA} und \citetitle{ZBA} mit empfohlenen Bereichen für die Krankheitsstufe statt fester Werte. Die offiziellen Angaben finden sich in \vref{tbl-suchtmittel}.\label{tbl-suchtmittel-revised}}
	\rowcolors{2}{tablelightblue}{white}
	\begin{tabular}{lc}
		\toprule
		{Name des Suchtmittels} & {Bereiche der Krankheitsstufe der Sucht} \\
		\hline
		Alkohol & 2~--~8 \\
		Boronwein & 6~--~12 \\
		Feuerschlick (Ilaris-Sucht) & 6~--~12 \\
		Libellengras & 1~--~8 \\
		Malomis & 10~--~20 \\
		Marbos Odem & 4~--~10 \\
		Mibelrohr & 4~--~10 \\
		Moarana & 2~--~8 \\
		Purpurmohn & 13 \\
		Regenbogenstaub & 6~--~12 \\
		Samthauch & 1~--~8 \\
		Schwarzer Wein & 5~--~14 \\
		Vierblättrige Einbeere & 5~--~14 \\
		Wachtrunk & 2~--~8 \\
		Wasserrausch & 1~--~8 \\
		Willenstrunk & 5~--~14 \\
		\bottomrule
	\end{tabular}
\end{table}

\paragraph{Entzug und minimale Konsumhäufigkeit}
Todo.

\paragraph{Generierungspunkte}
Um die Einschränkungen benannter Drogen zu berücksichtigen, gibt es für diese auch weiterhin GP in Höhe der doppelten Krankheitsstufe. Wer sich hingegen für allgemeine Wirkungen entscheidet, erhält nur GP in der Höhe der einfachen Krankheitsstufe.



