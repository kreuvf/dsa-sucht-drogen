\section{\enquote{Sucht} in DSA4.1}
Der Generierungsband \enquote{Wege der Helden} \cite{WdH} enthält den regeltechnischen Nachteil \enquote{Sucht} und definiert eine Sucht dabei wie folgt \cite[S.~271]{WdH}:

\begin{itemize}
	\item Die Generierungspunkte (GP), die man für die Wahl dieses Nachteils erhält, ergeben sich aus dem Doppelten der Giftstufe des Suchtmittels.
	\item Es gibt keinen Weg die Giftstufe zu berechnen. Stattdessen gibt es eine Liste sechs bekannter Suchtmittel mit den dazugehörigen festen Giftstufen\footnote{Während es bei den Suchtmitteln stets um die \enquote{Giftstufe} geht, wird bei der Sucht nach dem entsprechenden Suchtmittel von der \enquote{Krankheitsstufe der Sucht} gesprochen. Der Wert ist in beiden Fällen der gleiche.}. Im Alchemieband \enquote{Wege der Alchimie} \cite{WdA} und im Flora-und-Fauna-Band \enquote{Zoo-Botanica Aventurica} \cite{ZBA} finden sich zehn weitere Suchtmittel. Alle diese Suchtmittel mit den dazugehörigen Werten finden sich in [[Tabelle 1: Suchtmittelübersicht]]. Die Spannbreite der Giftstufen reicht von 5 bis 20, wobei man die beiden höchsten Werte von 13 und 20 aufgrund der Unspielbarkeit entsprechender Charaktere nicht zulassen sollte, was eine Spanne von 5 bis 12 lässt, also 10~GP bis 24~GP. Aufgrund einer Ausnahme ist eine Alkoholsucht nur die einfache Giftstufe in GP wert, womit die tatsächliche Spanne bei 6~GP bis 24~GP liegt.
	\item Der Drogenentzug läuft in zwei Stufen ab:
		\begin{itemize}
			\item Die erste Stufe beginnt nach Ablauf von 15 − Giftstufe Tagen, also für die Giftstufen zwischen 5 bis 12 nach 10 bis 3 Tagen.
			\item Die zweite Stufe beginnt nach Ablauf der jeweils doppelten Zeit, also 20 bis 6 Tagen.
			\item Die Symptome des Entzugs verschwinden jedoch immer mit der Einnahme einer Dosis, sofern bei der jeweiligen Droge nichts anderes geregelt ist, versteht sich.
		\end{itemize}
	\item Die regeltechnischen Auswirkungen des Entzugs:
		\begin{itemize}
			\item Fällt die Konstitution durch Überanstrengung unter 0, so erleidet der Süchtige Giftstufe/2 SP pro Tag, was im Schnitt die Regeneration aushebelt, die aber sowieso nicht mehr stattfindet.
			\item Auf Stufe 1 muss der Süchtige mit folgenden Einschränkungen leben: 
				\begin{itemize}
					\item Pro Tag Giftstufe/3 Erschöpfung, womit sich aufgrund der fehlenden Regeneration langsam, aber sicher das Erschöpfungskonto und danach das Überanstrengungskonto füllt. 
					\item Talentproben sind um Giftstufe/3 erschwert.
					\item Schlechte Eigenschaften sind um Giftstufe/3 erhöht.
				\end{itemize}
			\item Auf Stufe 2 sieht das Ganze ein wenig verschärfter aus: 
				\begin{itemize}
					\item Giftstufe/2 Erschöpfung pro Tag.
					\item Talentproben um Giftstufe/2 erschwert.
					\item Schlechte Eigenschaften um Giftstufe/2 erhöht.
				\end{itemize}
		\end{itemize}
\end{itemize}
