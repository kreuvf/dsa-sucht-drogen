% tail
\clearpage
\phantomsection
\printbibliography[title = {Quellen}, heading=bibnumbered]
\clearpage

\chapter{Hinweise}
\section{Kontakt}
Falls es Lob, Kritik oder Anmerkungen zu diesem Artikel gibt, freue ich mich jederzeit über eine \href{https://kreuvf.de/impressum.php}{E-Mail}. Wer einen Account bei GitHub hat, kann auch im Repository ein \emph{Issue} oder einen \emph{Pull Request} erstellen. Ich bitte allerdings um Zurückhaltung bei Pull Requests und empfehle, erst mit mir über geplante Änderungen/Erweiterungen zu sprechen, bevor unnötige Arbeit in einen Pull Request fließt, den ich dann aber nicht annehmen möchte.

\section{Danksagung}
Mein Dank gilt allen Testlesern für die Unterstützung mit wertvollen Hinweisen und Kommentaren während der Konzeptionsphase und auch nach der ersten Veröffentlichung.

\section{Lizenz}
Alle Rechte vorbehalten.
Dieser Artikel ist unter der Lizenz \enquote{\href{https://creativecommons.org/licenses/by/4.0/legalcode.de}{Creative Commons Namensnennung 4.0 International}} veröffentlicht. Es folgt eine rechtlich unverbindliche Zusammenfassung in allgemeinverständlicher Sprache:
\begin{itemize}
	\item Der Artikel darf in jedem Format oder Medium vervielfältigt und weiterverbreitet werden.
	\item Der Artikel darf verändert oder als Grundlage für eigene Werke genutzt werden. Der Zweck ist dabei nicht von Belang. Auch eine kommerzielle Verwertung ist erlaubt.
	\item Sie müssen angemessene Urheber- und Rechteangaben machen, einen Link auf die Lizenz anfügen und deutlich machen, ob Veränderungen vorgenommen wurden. Diese Angaben dürfen in jeder angemessenen Art und Weise gemacht werden, allerdings nicht so, dass der Eindruck entsteht, der Lizenzgeber unterstütze gerade Sie oder Ihre Nutzung besonders.
	\item Verändern Sie diesen Artikel oder benutzen Sie ihn als Grundlage für eigene Werke, so dürfen Sie den veränderten Artikel oder Ihr neues Werk nur unter denselben Bedingungen weitergeben, unter die Sie diesen Artikel nutzen dürfen.
	\item Sie dürfen keine zusätzlichen Klauseln oder technische Verfahren einsetzen, die anderen rechtlich irgendetwas untersagen, was die Lizenz erlaubt.
\end{itemize}

\section{Bildnachweis}
\begin{itemize}
	\item Seiten~\pageref{img-title} und \pageref{img-title-grey}: \href{https://pixabay.com/images/id-437262/}{Pixabay}, von \href{https://pixabay.com/users/stux-12364/}{stux} (gemeinfrei, bearbeitet)
	\item Seite~\pageref{img-herbs-title}: \href{https://pixabay.com/images/id-906140/}{Pixabay}, von \href{https://pixabay.com/users/kerdkanno-1334070/}{Seksak Kerdkanno} (gemeinfrei, bearbeitet)
	\item Seite~\pageref{img-white-lotus}: \href{https://commons.wikimedia.org/w/index.php?title=File:Nelumbo_lutea_blossom.jpeg&oldid=141771386}{Wikimedia Commons}, vom Natural Resources Conservation Service (gemeinfrei, bearbeitet)
	\item Seiten~\pageref{img-divider-a}, \pageref{img-divider-b}, \pageref{img-divider-c} und \pageref{img-divider-d}: \href{https://pixabay.com/images/id-4869416/}{Pixabay}, vom \href{https://pixabay.com/users/AnnaliseArt-7089643/}{AnnaliseArt} (gemeinfrei, bearbeitet)
	\item Alle Seiten außer den ersten beiden: \href{https://pixabay.com/images/id-1074131/}{Pixabay}, von \href{https://pixabay.com/users/ChrisFiedler-935884/}{ChrisFiedler} (gemeinfrei, bearbeitet)

\end{itemize}

\section{Stand\label{version-info}}
Dieses Dokument wird mit dem Versionsverwaltungsprogramm \href{https://git-scm.com/}{Git} verwaltet. Die Versionen dieses Dokuments können in einem \href{https://github.com/kreuvf/dsa-sucht-drogen}{Klon des Repositorys} gefunden werden. Die Version dieses Dokuments lautet \enquote{\gitAbbrevHash{}}\footnote{Die vollständige Versionsnummer lautet: \gitHash{}.} vom \gitAuthorIsoDate{} von \gitAuthorName{} (\href{mailto:webmaster@kreuvf.de}{webmaster@kreuvf.de}), der verwendete Branch ist \gitBranch.

% tail
