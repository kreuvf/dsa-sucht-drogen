\enquote{Sucht} ist ein regeltechnischer Nachteil, den Charaktere im System \enquote{Das Schwarze Auge 4.1} bei der Generierung erhalten können. Dazu gesellt sich eine Handvoll Rauschmittel, die meist teuer und aufgrund der Illegalität auch schwer zu beschaffen sind. Da ein Leben als Abenteurer allerdings häufig auch in entlegene Winkel Aventuriens führt, wird schnell ersichtlich, dass eine Sucht nicht mit einem Abenteurerleben in Einklang zu bringen und eine Sucht dadurch kaum spielbar ist. Dem Abhilfe zu schaffen, ist Ziel dieses Dokuments.

Der Nachteil lässt sich durch die Abschaffung fester Krankheitsstufen für unterschiedliche Suchtmittel und die Zulassung bevorzugter Rauschmittelwirkungen statt fester Rauschmittel flexibler gestalten. Auch an der Konsumhäufigkeit lässt sich drehen und bei den Auswirkungen eines Entzugs werden die offiziellen Regeln aus \enquote{Wege der Helden} denen aus \enquote{Zoo-Botanica Aventurica} vorgezogen.

Damit ist es allerdings noch nicht getan, denn so lässt sich zwar der Nachteil spielbarer gestalten, aber ein Süchtiger kann regelmäßig wiederkehrend einen guten Teil des Spielabends einfach nur dadurch einnehmen, dass er sich um die Beschaffung oder Herstellung seiner Rauschmittel kümmern muss. Dann folgt der Konsum, doch mit welchen Auswirkungen? Mittels detailliert ausgearbeiteter Probenmechanismen lässt sich das Spiel mit der Sucht in wenigen Minuten und Würfen abhandeln. Als Bonus für den Meister bieten sich gerade bei gescheiterten Beschaffungsproben viele Anknüpfpunkte für das Abenteuer.
