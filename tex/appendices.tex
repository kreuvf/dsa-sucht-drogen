% appendices
\appendix

\chapter{Tabellen}
\epigraph{Wenn es DSA an etwas mangelt, dann sind es definitiv Tabellen.}{\emph{--- Niemand jemals}}
% Use \vspace*{-\baselineskip} to circumvent strut after itemize giving a blank line
% Source: https://tex.stackexchange.com/a/141674/92368
{
\phantomsection
\addcontentsline{toc}{section}{Auswirkungen misslungener Rauschmittelbeschaffung in der Unterwelt}
\setlist[itemize]{leftmargin=12pt}
\rowcolors{2}{tablelightblue}{white}
\begin{tabularx}{\linewidth}{p{1.2cm}p{14.8cm}}
	\caption[Auswirkungen misslungener Rauschmittelbeschaffung in der Unterwelt]{Auswirkungen misslungener Rauschmittelbeschaffung in der Unterwelt. Misslungene Gassenwissenproben zur Beschaffung von Rauschmitteln in der Unterwelt können bestimmte Auswirkungen haben, die über den unter \vref{beschaffung-unterwelt} beschriebenen Mechanismus ein Ergebnis von mindestens 2 liefern. Der Schweregrad der Auswirkung steigt mehr oder weniger mit dem Ergebnis an. Da etliche Auswirkungen auch zu Schadenspunkten führen, sei dem Meister ans Herz gelegt, dass diese Tabelle \emph{nicht} das Ziel verfolgt, dem Süchtigen möglichst den Spielspaß zu vermiesen und ihn ins Grab zu führen. Kein Charakter soll durch die bloße Benutzung dieser Auswirkungentabelle befürchten müssen zu sterben. Sollte eine bestimmte Auswirkung unpassend erscheinen, liegt es am Meister sich in der Nähe des Ergebnisses umzuschauen, um eine geeignetere Auswirkung zu finden.\label{tbl-auswirkungen}} \\
	\toprule
	{\cellcolor{white}Ergebnis} & {\cellcolor{white}Auswirkung} \\
	\hline \endfirsthead
	\caption[]{\cellcolor{white}\textit{Fortsetzung von der vorhergehenden Seite (Tabellenanfang auf S.~\pageref{tbl-auswirkungen})}} \\
	\toprule
	{\cellcolor{white}Ergebnis} & {\cellcolor{white}Auswirkung} \\
	\hline
	\endhead
	\hline
	\multicolumn{2}{r}{\cellcolor{white}\textit{Fortsetzung auf der nächsten Seite}} \\
	\endfoot
	\bottomrule
	\endlastfoot
	\TablesafeInputIfFileExists{res/consequences/consequences.tex}{}{}
\end{tabularx}
}

\afterpage{
	\clearpage
	\begin{landscape}
		\addcontentsline{toc}{section}{Auswirkungen des Rauschmittelkonsums}
		\rowcolors{2}{tablelightblue}{white}
		\setlength{\tabcolsep}{4.25pt}
		\setlist[itemize]{leftmargin=12pt}
		\setlist[enumerate]{leftmargin=12pt}
		\begin{tabularx}{\textheight}{@{\hspace{1pt}}C{1.7cm}@{\hspace{2pt}}C{1.1cm}@{\hspace{2pt}}L{4.0cm}C{1.3cm}L{12.6cm}C{0.35cm}@{\hspace{4pt}}C{0.35cm}@{\hspace{4pt}}C{0.35cm}@{\hspace{4pt}}C{0.35cm}@{\hspace{1pt}}}
			\caption[Auswirkungen des Rauschmittelkonsums]{Auswirkungen des Rauschmittelkonsums. Aus den Würfen der ersten beiden Spalten lässt sich die dazugehörige Auswirkung des Rauschmittelkonsums ablesen. Magische Charaktere verwenden für den ersten Wurf 1W4, alle anderen Charaktere 1W3. Ist eine Auswirkung noch nicht definiert, so muss der W20 erneut gewürfelt werden. Ist mit der Auswirkung ein Zufallswert verbunden, muss dieser ebenfalls ausgewürfelt werden. Für die Auswirkung können weitere Würfe nötig werden. Die letzten vier Spalten geben die Empfehlung für die Eignung einer Auswirkung für eine bestimmte Nebenwirkungsklasse an.\label{tbl-nebenwirkungen}} \\
			\toprule
			{\cellcolor{white}1W3/1W4} & {\cellcolor{white}1W20} & {\cellcolor{white}Kurzbeschreibung} & {\cellcolor{white}Wert} & {\cellcolor{white}Regeltechnische Auswirkungen} &
				{\cellcolor{white}K} & {\cellcolor{white}M} & {\cellcolor{white}L} & {\cellcolor{white}P} \\
			\hline \endfirsthead
			\caption[]{\cellcolor{white}\textit{Fortsetzung von der vorhergehenden Seite (Tabellenanfang auf S.~\pageref{tbl-nebenwirkungen})}} \\
			\toprule
			{\cellcolor{white}1W3/1W4} & {\cellcolor{white}1W20} & {\cellcolor{white}Kurzbeschreibung} & {\cellcolor{white}Wert} & {\cellcolor{white}Regeltechnische Auswirkungen} &
				{\cellcolor{white}K} & {\cellcolor{white}M} & {\cellcolor{white}L} & {\cellcolor{white}P} \\
			\hline
			\endhead
			\hline
			\multicolumn{9}{r}{\cellcolor{white}\textit{Fortsetzung auf der nächsten Seite}} \\
			\endfoot
			\bottomrule
			\endlastfoot
			\TablesafeInputIfFileExists{res/sideeffects.tex}{}{}
		\end{tabularx}
	\end{landscape}
	\clearpage
}

\chapter{Schnellreferenz\label{quick-ref}}
Diese Schnellreferenz enthält alle Regeln zum Spiel mit Süchten und Rauschmitteln in kompakter Form.

\begin{multicols}{2}
\setlist[itemize]{leftmargin=12pt}
\section*{Generierung}
Entweder Sucht nach benannten Suchtmitteln (\vref{tbl-suchtmittel-revised}) für Generierungspunkte in Höhe der doppelten Stufe der Sucht oder unbenannte Suchtmittel (Wirkung mit Meister festlegen) für Generierungspunkte in Höhe der Stufe der Sucht wählen.

\section*{Beschaffung}
\subsection*{Beschaffung in der Unterwelt}
\begin{itemize}[nosep]
	\item \emph{Gassenwissen}-Probe + halbe Stufe der Sucht
	\item Doppel-1: Qualität: TaP*~+~4W3 (→~Konsum)
	\item Gelungen: Qualität: TaP* (→~Konsum)
	\item Misslungen: 1W20 gegen |~TaP*~| (→~Gelungen: |~TaP*~|~+~1W20! in \vref{tbl-auswirkungen} nachschlagen; Misslungen: keine Konsequenzen)
	\item Doppel-20: |~TaP*~|~+~5~+~1W20! in \vref{tbl-auswirkungen}
\end{itemize}
\subsection*{Herstellung auf eigene Faust}
\paragraph*{Beschaffung von Ausgangsmaterial}
\begin{itemize}[nosep]
	\item \emph{Kräutersuchen}-Probe + halbe Stufe der Sucht
	\item Doppel-1: 1~+~1W3~+~1~pro halbe Stufe der Sucht TaP* Portionen der Qualität TaP*~+~4W3 (→~Herstellung)
	\item Gelungen: 1~+~1~pro halbe Stufe der Sucht TaP* Portionen der Qualität TaP* (→~Herstellung)
	\item Misslungen: kein Ausgangsmaterial
	\item Doppel-20: 1~+~1~pro halbe Stufe der Sucht TaP* Portionen der Qualität −|~TaP*~|~−~4W3 (→~Herstellung)
\end{itemize}
\paragraph*{Herstellung der Rauschmittel}
Es wird mindestens ein archaisches Labor benötigt.
\begin{itemize}[nosep]
	\item \emph{Alchemie}-Probe (Rauschmittel) oder \emph{Kochen}-Probe (Tränke) +~halbe Stufe der Sucht −~Qualität der Ausgangsmaterialien
	\item Doppel-1: 1~Anwendung pro Portion der Qualität TaP*~+~4W3
	\item Gelungen: 1~+~1~pro halbe Stufe der Sucht TaP* Portionen (maximal Anzahl Portionen der Ausgangsstoffe)	der Qualität TaP*
	\item Misslungen: keine Rauschmittel
	\item Doppel-20: 1~+~1~pro halbe Stufe der Sucht TaP* Portionen (auch über die Anzahl Portionen der Ausgangsstoffe hinaus) der Qualität −|~TaP*~|~−~4W3 
\end{itemize}

\section*{Konsum}
\begin{itemize}[nosep]
	\item \emph{Zechen}-Probe (Spezialisierung) +~halbe Stufe der Sucht −~Qualität der Drogen
	\item Eintreten der Hauptwirkung und gegebenenfalls noch Prüfwurf für Nebenwirkungen nach \vref{tbl-auswirkungen-konsum}, Beginn und Ende der Nebenwirkungen nach \vref{tbl-nebenwirkungsklassen}
	\item Nebenwirkungen inklusive aller Werte entsprechend \vref{tbl-nebenwirkungen} bestimmen
	\item Konsumpause nach \vref{tbl-konsum} bestimmen (verdeckt durch den Meister)
\end{itemize}
\end{multicols}

\end{document}

% appendices
