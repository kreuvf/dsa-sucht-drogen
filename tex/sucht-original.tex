\chapter{Regeln zu Sucht und Rauschmitteln}
Der Generierungsband \citetitle{WdH} enthält den regeltechnischen Nachteil \emph{Sucht} und definiert eine Sucht dabei wie folgt \cite[S.~271]{WdH}:

\begin{itemize}
	\item Die Generierungspunkte (GP), die man für die Wahl dieses Nachteils erhält, ergeben sich aus dem Doppelten der Giftstufe des Suchtmittels.
	\item Es gibt keinen Weg die Giftstufe zu berechnen. Stattdessen gibt es eine Liste sechs bekannter Suchtmittel mit den dazugehörigen festen Giftstufen\footnote{Während es bei den Suchtmitteln stets um die \enquote{Giftstufe} geht, wird bei der Sucht nach dem entsprechenden Suchtmittel von der \enquote{Krankheitsstufe der Sucht} gesprochen. Der Wert ist in beiden Fällen der gleiche.}. Im Alchemieband \citetitle{WdA} und im Flora-und-Fauna-Band \citetitle{ZBA} finden sich zehn weitere Suchtmittel. Alle diese Suchtmittel mit den dazugehörigen Werten finden sich in \vref{tbl-suchtmittel}. Die Spannbreite der Giftstufen reicht von 5 bis 20, wobei man die beiden höchsten Werte von 13 und 20 aufgrund der Unspielbarkeit entsprechender Charaktere nicht zulassen sollte, was eine Spanne von 5 bis 12 lässt, also \SIGP{10} bis \SIGP{24}. Aufgrund einer Ausnahme ist eine Alkoholsucht nur die einfache Giftstufe in GP wert, womit die tatsächliche Spanne bei \SIGP{6} bis \SIGP{24} liegt.
	\item Der Rauschmittelentzug läuft in zwei Stufen ab\footnote{Im \citetitle{WdS} finden sich die gleichen Entzugsregeln wie im \citetitle{WdH}, in der \citetitle{ZBA} \cite[S.~225]{ZBA} davon abweichende Regeln, die ich hier nicht weiter kommentieren möchte.}, sofern beim Suchtmittel nichts anderes angegeben ist:
		\begin{itemize}
			\item Die erste Stufe beginnt nach Ablauf von 15 − Giftstufe Tagen, also für die Giftstufen zwischen 5 bis 12 nach 10 bis 3 Tagen.
			\item Die zweite Stufe beginnt nach Ablauf der jeweils doppelten Zeit, also 20 bis 6 Tagen.
			\item Die Symptome des Entzugs verschwinden jedoch immer mit der Einnahme einer Dosis, sofern beim jeweiligen Rauschmittel nichts anderes geregelt ist, versteht sich.
		\end{itemize}
	\item Die regeltechnischen Auswirkungen des Entzugs:
		\begin{itemize}
			\item Fällt die Konstitution durch Überanstrengung unter 0, so erleidet der Süchtige Giftstufe/2 SP pro Tag, was im Schnitt die Regeneration aushebelt, die aber sowieso nicht mehr stattfindet.
			\item Auf Stufe 1 muss der Süchtige mit folgenden Einschränkungen leben: 
				\begin{itemize}
					\item Pro Tag Giftstufe/3 Erschöpfung, womit sich aufgrund der fehlenden Regeneration langsam, aber sicher das Erschöpfungskonto und danach das Überanstrengungskonto füllt. 
					\item Talentproben sind um Giftstufe/3 erschwert.
					\item Schlechte Eigenschaften sind um Giftstufe/3 erhöht.
				\end{itemize}
			\item Auf Stufe 2 sieht das Ganze ein wenig verschärfter aus: 
				\begin{itemize}
					\item Giftstufe/2 Erschöpfung pro Tag.
					\item Talentproben um Giftstufe/2 erschwert.
					\item Schlechte Eigenschaften um Giftstufe/2 erhöht.
				\end{itemize}
		\end{itemize}
\end{itemize}

\begin{table}
	\centering
	\caption[Suchtmittel aus \citetitle{WdH}, \citetitle{WdA} und \citetitle{ZBA}]{Übersicht aller Suchtmittel aus \citetitle{WdH}, \citetitle{WdA} und \citetitle{ZBA} mit den ihnen zugeordneten Krankheitsstufen der Sucht. Neben den hier aufgelisteten Suchtmitteln gibt es noch weitere Rauschmittel wie Tabak, Cheriacha oder Zithabar, die allerdings keine regeltechnische Sucht auslösen. \cite[S.~271]{WdH}\label{tbl-suchtmittel}}
	\rowcolors{2}{tablelightblue}{white}
	\begin{tabular}{lcr}
		\toprule
		{Name des Suchtmittels} & {Krankheitsstufe der Sucht} & {Quellen} \\
		\hline
		Alkohol & 6 & \cite[S.~271]{WdH} \\
		Boronwein & 12 & \cite[S.~271]{WdH}\cite[S.~272]{ZBA} \\
		Feuerschlick (Ilaris-Sucht) & 12 & \cite[S.~237]{ZBA} \\
		Libellengras & 5 & \cite[S.~194]{WdA} \\
		Malomis & 20 & \cite[S.~248]{ZBA} \\
		Marbos Odem & 8 & \cite[S.~271]{WdH}\cite[S.~247]{ZBA} \\
		Mibelrohr & 8 & \cite[S.~251]{ZBA} \\
		Moarana & 6 & \cite[S.~271]{WdH}\cite[S.~201]{ZBA} \\
		Purpurmohn & 13 & \cite[S.~253]{ZBA} \\
		Regenbogenstaub & 12 & \cite[S.~63]{WdA} \\
		Samthauch & 5 & \cite[S.~271]{WdH}\cite[S.~266]{ZBA} \\
		Schwarzer Wein & 10 & \cite[S.~271]{WdH} \\
		Vierblättrige Einbeere & 10 & \cite[S.~271]{ZBA} \\
		Wachtrunk & 6 & \cite[S.~64]{WdA} \\
		Wasserrausch & 5 & \cite[S.~273]{ZBA} \\
		Willenstrunk & 10 & \cite[S.~43/44]{WdA} \\
		\bottomrule
	\end{tabular}
\end{table}

\section{Beschaffung, Herstellung, Auswirkungen~--~Regelungslücken?}
Während der Nachteil vor allem für die Erstellung des Charakters wichtig ist, stellt sich die Frage, inwieweit das Ausspielen der Sucht am Spieltisch durch den Nachteil abgedeckt wird.

\subsection{Beschaffung}
Während das legale Rauschmittel Alkohol problemlos verfügbar ist\footnote{Die Legalität und leichte Verfügbarkeit des Alkohols gehen sogar so weit, dass es für eine Alkoholsucht nur die halbe Menge an Generierungspunkten gibt. Dies ist insoweit fraglich, als vierblättrige Einbeeren oder Willenstrünke ebenfalls leicht verfügbar und weit verbreitet sind, aber die vollen Generierungspunkte geben.}, lassen sich andere nur auf Umwegen organisieren. Entweder muss man sein Glück im Untergrund versuchen oder das Rauschmittel in Eigenregie herstellen.

Leider gibt es keine Regeln zur Beschaffung, weshalb der Meister keine Anhaltspunkte für die Bestimmung der notwendigen Proben und der heranzuziehenden Modifikatoren hat.

Ein weiterer wichtiger Punkt ist die lokale Verfügbarkeit des Suchtmittels. Aufgrund der eingeschränkten Legalität vieler Suchtmittel ist der Schmuggel über weite Distanzen risikoreich und teuer~--~entsprechend kostet ein ohnehin schon kaum erschwingliches vor Ort produziertes Rauschmittel fernab der Heimat ein kleines Vermögen. Wer sogar das Pech hat, dass ihn Abenteuer in tiefe Wälder, Gebirge oder Höhlen führen\footnote{Bei vielen Abenteuern ist es gang und gäbe nach einem Auftakt an einem bekannteren Ort eine Reise zu abgelegenen Orten anzutreten. Das liegt auch schlicht daran, dass die bekannteren Orte bereits detailliert beschrieben sind, der Abenteuerautor also kaum Gestaltungsspielraum in einer großen Stadt hat. Will heißen: Diese Situation kommt möglicherweise häufiger vor als man im ersten Moment denkt.}, also an Orte, die ohnehin schon keine Fernhändler sehen, muss sich entweder vorher eindecken oder mit den Entzugssymptomen klarkommen. Ersteres ist teuer und auffällig, Zweiteres frustrierend für den Spieler. Eine Möglichkeit auf andere Suchtmittel auszuweichen, ist nicht vorgesehen.

\subsection{Herstellung}
Bei der Herstellung sieht es da schon anders aus. Für alle in \citetitle{WdA} genannten alchimistischen Suchtmittel gibt es Herstellregeln, allerdings weichen die Entzugserscheinungen dafür teils so drastisch vom Standard ab, dass selbst reiche Charaktere nicht in der Lage wären die entsprechenden Kosten aufzubringen: beispielsweise mindestens \SID{50} pro Tag für Regenbogenstaub. Auch die Herstellung der in \citetitle{ZBA} genannten pflanzlichen Suchtmittel ist geregelt, wobei sowohl Purpurmohn (\SIGP{26}) als auch Malomis (\SIGP{40}) quasi-unspielbare Charaktere erzeugen.

\subsection{Auswirkung}
Die Auswirkungen der meisten Suchtmittel sind detailliert beschrieben. Allerdings wird dabei stets von reinen Suchtmitteln ausgegangen, die nicht mit irgendwelchen Zusätzen gestreckt wurden und neben der beabsichtigten Hauptwirkung wechselnde Nebenwirkungen aufweisen können. Für solche gepanschten Suchtmittel gibt es für Meister keinerlei Handreichungen.

\subsection{Fazit}
Die Beschaffung der meisten Suchtmittel ist nicht offiziell geregelt, Anhaltspunkte zur Qualität der aus dunklen Kanälen bezogenen Rauschmittel fehlen ebenso. Wie die Beschaffung nur lokal hergestellter Rauschmittel fernab des Kernverbreitungsgebietes ausschaut, ist nicht festgelegt. Ein Ausweichen auf andere Rauschmittel, die dafür aber lokal vorhanden sind, ist zwar denkbar, aber nicht beschrieben. Sowohl zur Herstellung als auch zu den Wirkungen, nicht aber möglichen Nebenwirkungen, gibt es genügend Details.

Es fehlt also vor allem eine regelseitige Ausgestaltung der Beschaffung, des Ausweichens auf alternative Rauschmittel und möglicher Nebenwirkungen.

